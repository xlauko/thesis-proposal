% ****************************************************************************************************
% classicthesis-config.tex
% formerly known as loadpackages.sty, classicthesis-ldpkg.sty, and classicthesis-preamble.sty
% Use it at the beginning of your ClassicThesis.tex, or as a LaTeX Preamble
% in your ClassicThesis.{tex,lyx} with % ****************************************************************************************************
% classicthesis-config.tex
% formerly known as loadpackages.sty, classicthesis-ldpkg.sty, and classicthesis-preamble.sty
% Use it at the beginning of your ClassicThesis.tex, or as a LaTeX Preamble
% in your ClassicThesis.{tex,lyx} with % ****************************************************************************************************
% classicthesis-config.tex
% formerly known as loadpackages.sty, classicthesis-ldpkg.sty, and classicthesis-preamble.sty
% Use it at the beginning of your ClassicThesis.tex, or as a LaTeX Preamble
% in your ClassicThesis.{tex,lyx} with % ****************************************************************************************************
% classicthesis-config.tex
% formerly known as loadpackages.sty, classicthesis-ldpkg.sty, and classicthesis-preamble.sty
% Use it at the beginning of your ClassicThesis.tex, or as a LaTeX Preamble
% in your ClassicThesis.{tex,lyx} with \input{classicthesis-config}
% ****************************************************************************************************
% If you like the classicthesis, then I would appreciate a postcard.
% My address can be found in the file ClassicThesis.pdf. A collection
% of the postcards I received so far is available online at
% http://postcards.miede.de
% ****************************************************************************************************
\PassOptionsToPackage{utf8}{inputenc}
\usepackage{inputenc}

\usepackage[dvipsnames]{xcolor}  % Coloured text etc.
\definecolor{orioles}{HTML}{F6511D}
\definecolor{ucla}{HTML}{FFB400}
\definecolor{vivid}{HTML}{00A6ED}
\definecolor{apple}{HTML}{7FB800}
\definecolor{pruss}{HTML}{0D2C54}

% ****************************************************************************************************
% 1. Configure classicthesis for your needs here, e.g., remove "drafting" below
% in order to deactivate the time-stamp on the pages
% ****************************************************************************************************
\PassOptionsToPackage{
  drafting=false,    % print version information on the bottom of the pages
  tocaligned=false, % the left column of the toc will be aligned (no indentation)
  dottedtoc=true,  % page numbers in ToC flushed right
  eulerchapternumbers=true, % use AMS Euler for chapter font (otherwise Palatino)
  linedheaders=false,       % chaper headers will have line above and beneath
  floatperchapter=true,     % numbering per chapter for all floats (i.e., Figure 1.1)
  eulermath=true,  % use awesome Euler fonts for mathematical formulae (only with pdfLaTeX)
  beramono=true,    % toggle a nice monospaced font (w/ bold)
  palatino=false,    % deactivate standard font for loading another one, see the last section at the end of this file for suggestions
  style=classicthesis % classicthesis, arsclassica
}{classicthesis}

% ****************************************************************************************************
% 2. Personal data and user ad-hoc commands
% ****************************************************************************************************
\newcommand{\Title}{Abstractions via Program Transformations}

\newcommand{\Subtitle}{phd thesis proposal\xspace}
\newcommand{\Logo}{img/fi-logo}
\newcommand{\Author}{Henrich Lauko\xspace}
\newcommand{\Supervisor}{prof. RNDr. Jiří Barnat, Ph.D.\xspace}
\newcommand{\Consultant}{RNDr. Petr Ročkai, Ph.D.\xspace}
\newcommand{\Department}{Faculty of Informatics\xspace}
\newcommand{\University}{Masaryk University\xspace}
\newcommand{\Year}{September 2019\xspace}
\newcommand{\myVersion}{version 0.1\xspace}

% ********************************************************************
% Setup, finetuning, and useful commands
% ********************************************************************
\newcounter{dummy} % necessary for correct hyperlinks (to index, bib, etc.)
\newlength{\abcd} % for ab..z string length calculation
\providecommand{\mLyX}{L\kern-.1667em\lower.25em\hbox{Y}\kern-.125emX\@}
\newcommand{\ie}{i.\,e.}
\newcommand{\Ie}{I.\,e.}
\newcommand{\eg}{e.\,g.}
\newcommand{\Eg}{E.\,g.}

% ****************************************************************************************************
% 3. Loading some handy packages
% ****************************************************************************************************
% ********************************************************************
% Packages with options that might require adjustments
% ********************************************************************
%\PassOptionsToPackage{ngerman,american}{babel}   % change this to your language(s)
% Spanish languages need extra options in order to work with this template
%\PassOptionsToPackage{spanish,es-lcroman}{babel}
\usepackage{babel}
\usepackage{adjustbox}

\usepackage{enumitem}

\usepackage{minted}
%\usepackage{inconsolata}
\setminted{
    frame=lines,
    rulecolor=pruss,
    framesep=1mm,
    autogobble,
    numbersep=8pt,
    baselinestretch=1.2,
    fontsize=\footnotesize,
    escapeinside=||,
    mathescape=true
}
\usemintedstyle{lovelace}

\newmintinline[llvmint]{llvm}{}

\newcommand{\code}[1]{\texttt{#1}}
\newcommand{\ccode}[1]{\mintinline{cpp}{#1}}
\newcommand{\lcode}[1]{\mintinline{llvm}{#1}}

\definecolor{varcolor}{HTML}{B83838}
\newcommand{\var}[1]{\hspace{-0.5em}\color{varcolor}\text{#1}}

\usepackage{csquotes}

\PassOptionsToPackage{%
    %backend=biber, %instead of bibtex
	backend=bibtex8,bibencoding=ascii,%
	language=auto,%
	style=alphabetic,%
        citestyle=alphabetic,
        firstinits=true,
    %style=authoryear-comp, % Author 1999, 2010
    %bibstyle=authoryear,dashed=false, % dashed: substitute rep. author with ---
    sorting=nyt, % name, year, title
    maxbibnames=10, % default: 3, et al.
    %backref=true,%
    natbib=true % natbib compatibility mode (\citep and \citet still work)
}{biblatex}
\usepackage{biblatex}

\renewcommand\mkbibnamefamily[1]{\textsc{#1}}

\newcommand{\prule}{
    {\color{pruss} \hrule}%
}

\PassOptionsToPackage{fleqn}{amsmath} % math environments and more by the AMS
\usepackage{amsmath,amsfonts,amssymb,amsthm}

\usepackage{framed,  etoolbox}
  \colorlet{framecolor}{pruss!40}
\usepackage{thmtools}

\makeatletter
\define@key{thmdef}{frame}[{}]{%
 \thmt@trytwice{}{%
 \RequirePackage{framed}%
 \RequirePackage{thm-patch}%
 \addtotheorempreheadhook[\thmt@envname]{\medskip \par}%
 \addtotheorempostfoothook[\thmt@envname]{\medskip}%
 }%
}
\makeatother

\declaretheorem[numberwithin=section, frame]{definition}
\declaretheorem[numberwithin=section, frame]{example}

\DeclareSymbolFont{operators}{\encodingdefault}{ppl}{m}{n}
\DeclareMathAlphabet{\mathbf}{\encodingdefault}{ppl}{bx}{n}
\DeclareMathAlphabet{\mathit}{\encodingdefault}{ppl}{m}{it}

% ********************************************************************
% General useful packages
% ********************************************************************
\PassOptionsToPackage{T1}{fontenc} % T2A for cyrillics

%\usepackage{fontspec}
\usepackage{textcomp} % fix warning with missing font shapes
\usepackage{scrhack} % fix warnings when using KOMA with listings package
\usepackage{xspace} % to get the spacing after macros right
\usepackage{mparhack} % get marginpar right
\usepackage{marginnote}
\PassOptionsToPackage{printonlyused,smaller}{acronym}
  \usepackage{acronym} % nice macros for handling all acronyms in the thesis
  %\renewcommand{\bflabel}[1]{{#1}\hfill} % fix the list of acronyms --> no longer working
  %\renewcommand*{\acsfont}[1]{\textsc{#1}}
  %\renewcommand*{\aclabelfont}[1]{\acsfont{#1}}
  %\def\bflabel#1{{#1\hfill}}
  \def\bflabel#1{{\acsfont{#1}\hfill}}
  \def\aclabelfont#1{\acsfont{#1}}

\renewcommand*{\aclabelfont}[1]{\acsfont{#1}}

\usepackage{microtype}

\clubpenalty  = 8000        % help suppress orphans, default = 4,000 (?), from 0 to 10 000 (from 300 to 1 000 recommended, 10 000 not recommended)
\widowpenalty = 8000        % help suppress widows,  default = 4,000 (?), from 0 to 10 000 (from 300 to 1 000 recommended, 10 000 not recommended)
\SetTracking{encoding={*}, shape=sc}{40}

\usepackage{syntax}

% ****************************************************************************************************
% 4. Setup floats: tables, (sub)figures, and captions
% ****************************************************************************************************
\usepackage[strict]{changepage}%
\usepackage{colortbl}
\usepackage{booktabs,tabularx}%
    \setlength{\extrarowheight}{3pt} % increase table row height
\newcommand{\tableheadline}[1]{\multicolumn{1}{c}{\spacedlowsmallcaps{#1}}}
\newcommand{\myfloatalign}{\centering} % to be used with each float for alignment
\usepackage{caption}
\captionsetup{format=plain,font=small, textfont=it, labelfont=bf}
\usepackage{subfig}

\newcolumntype{g}{>{\columncolor{pruss!10}}c}

\usepackage{xkeyval}
\usepackage{subfig}
\usepackage{graphicx}
\usepackage{calc}

% ****************************************************************************************************
% 6. PDFLaTeX, hyperreferences and citation backreferences
% ****************************************************************************************************
\PassOptionsToPackage{xetex,hyperfootnotes=false,pdfpagelabels}{hyperref}
\usepackage{hyperref}  % backref linktocpage pagebackref
%\pdfcompresslevel=9
%\pdfadjustspacing=1
\PassOptionsToPackage{xetex}{graphicx}
\usepackage{graphicx}
\usepackage{pdfpages}

% ********************************************************************
% Hyperreferences
% ********************************************************************
\hypersetup{%
    %draft, % = no hyperlinking at all (useful in b/w printouts)
    colorlinks=true, linktocpage=true, pdfstartpage=3, pdfstartview=FitV,%
    % uncomment the following line if you want to have black links (e.g., for printing)
    %colorlinks=false, linktocpage=false, pdfstartpage=3, pdfstartview=FitV, pdfborder={0 0 0},%
    breaklinks=true, pdfpagemode=UseNone, pageanchor=true, pdfpagemode=UseOutlines,%
    plainpages=false, bookmarksnumbered, bookmarksopen=true, bookmarksopenlevel=1,%
    hypertexnames=true, pdfhighlight=/O,%nesting=true,%frenchlinks,%
    urlcolor=webbrown, linkcolor=pruss, citecolor=pruss, %pagecolor=RoyalBlue,%
    %urlcolor=Black, linkcolor=Black, citecolor=Black, %pagecolor=Black,%
    pdftitle={\Title},%
    pdfauthor={\textcopyright\ \Author, \University, \Department},%
    pdfsubject={},%
    pdfkeywords={},%
    pdfcreator={pdfLaTeX},%
    pdfproducer={LaTeX with hyperref and classicthesis}%
}

% ********************************************************************
% Setup autoreferences
% ********************************************************************
\makeatletter
\@ifpackageloaded{babel}%
    {%
       \addto\extrasamerican{%
			\renewcommand*{\figureautorefname}{Figure}%
			\renewcommand*{\tableautorefname}{Table}%
			\renewcommand*{\partautorefname}{Part}%
			\renewcommand*{\chapterautorefname}{Chapter}%
			\renewcommand*{\sectionautorefname}{Section}%
			\renewcommand*{\subsectionautorefname}{Section}%
			\renewcommand*{\subsubsectionautorefname}{Section}%
                }%
       \addto\extrasngerman{%
			\renewcommand*{\paragraphautorefname}{Absatz}%
			\renewcommand*{\subparagraphautorefname}{Unterabsatz}%
			\renewcommand*{\footnoteautorefname}{Fu\"snote}%
			\renewcommand*{\FancyVerbLineautorefname}{Zeile}%
			\renewcommand*{\theoremautorefname}{Theorem}%
			\renewcommand*{\appendixautorefname}{Anhang}%
			\renewcommand*{\equationautorefname}{Gleichung}%
			\renewcommand*{\itemautorefname}{Punkt}%
                }%
            % Fix to getting autorefs for subfigures right (thanks to Belinda Vogt for changing the definition)
            \providecommand{\subfigureautorefname}{\figureautorefname}%
    }{\relax}
\makeatother


% ********************************************************************
% Development Stuff
% ********************************************************************
\listfiles

% ********************************************************************
% Last, but not least...
% ********************************************************************
\usepackage{classicthesis}

\areaset[current]{312pt}{657pt}
\setlength{\marginparwidth}{9.5em}%
\setlength{\marginparsep}{2em}%

\usepackage{xargs}
\usepackage[colorinlistoftodos,prependcaption,textsize=tiny]{todonotes}
\newcommandx{\add}[2][1=]{\todo[inline, linecolor=red,backgroundcolor=red!25,bordercolor=red,#1]{#2}}
\newcommandx{\change}[2][1=]{\todo[inline,linecolor=blue,backgroundcolor=blue!25,bordercolor=blue,#1]{#2}}
\newcommandx{\info}[2][1=]{\todo[inline,linecolor=OliveGreen,backgroundcolor=OliveGreen!25,bordercolor=OliveGreen,#1]{#2}}
\newcommandx{\improvement}[2][1=]{\todo[inline,linecolor=Plum,backgroundcolor=Plum!25,bordercolor=Plum,#1]{#2}}
\newcommandx{\thiswillnotshow}[2][1=]{\todo[disable,#1]{#2}}

\makeatletter
\newcommand*{\textoverline}[1]{$\overline{\hbox{#1}}\m@th$}
\makeatother

\usepackage{tikz-picture}

% ****************************************************************************************************
% 8. Further adjustments (experimental)
% ****************************************************************************************************
% ********************************************************************
% Changing the text area
% ********************************************************************
\linespread{1.05} % a bit more for Palatino
\areaset[current]{312pt}{720pt} % 686 (factor 2.2) + 33 head + 42 head \the\footskip
\setlength{\marginparwidth}{9em}%
\setlength{\marginparsep}{2em}%

% ********************************************************************
% Using different fonts
% ********************************************************************
%\setmathfont{MnSymbol}
%\setmathfont[range=\mathup/{num,latin,Latin,greek,Greek}]{Minion Pro}
%\setmathfont[range=\mathbfup/{num,latin,Latin,greek,Greek}]{MinionPro-Bold}
%\setmathfont[range=\mathit/{num,latin,Latin,greek,Greek}]{MinionPro-It}
%\setmathfont[range=\mathbfit/{num,latin,Latin,greek,Greek}]{MinionPro-BoldIt}
%\setmathrm{Minion Pro}

%\usepackage{unicode-math}

%\setmainfont[Numbers={OldStyle,Proportional},SmallCapsFeatures={LetterSpace=6}]{Minion Pro}

%\usepackage[oldstylenums]{kpfonts} % oldstyle notextcomp
%\usepackage[osf]{libertine}
%\usepackage[light,condensed,math]{iwona}
%\renewcommand{\sfdefault}{iwona}
%\usepackage{lmodern} % <-- no osf support :-(
%\usepackage{cfr-lm} %
%\usepackage[urw-garamond]{mathdesign} %<-- no osf support :-(
%\usepackage[default,osfigures]{opensans} % scale=0.95
%\usepackage[sfdefault]{FiraSans}

\usepackage{pifont}

%\usepackage[math-style=upright]{unicode-math}
\usepackage{unicode-math}

\setmainfont[Numbers={OldStyle,Proportional},SmallCapsFeatures={LetterSpace=6}]{Minion Pro}
\setmathfont[Scale=MatchLowercase]{STIX2Math.otf}
\setmonofont[Scale=MatchLowercase]{Inconsolata}

\usepackage{polyglossia}
\setmainlanguage{english}
\setotherlanguage{czech}

\usepackage{tocloft}
\renewcommand{\cftchapfont}{\normalfont\scshape}
\renewcommand{\cftpartfont}{\color{CTtitle}\normalfont\scshape}%
\setlength{\cftbeforechapskip}{7pt}
%\renewcommand\cftsecafterpnum{\vskip15pt}

% ****************************************************************************************************

\usepackage{sidenotes}

\newenvironment{summary}
{\bigskip\hrule\bigskip\itshape}
{\bigskip\hrule}

% ****************************************************************************************************
% If you like the classicthesis, then I would appreciate a postcard.
% My address can be found in the file ClassicThesis.pdf. A collection
% of the postcards I received so far is available online at
% http://postcards.miede.de
% ****************************************************************************************************
\PassOptionsToPackage{utf8}{inputenc}
\usepackage{inputenc}

\usepackage[dvipsnames]{xcolor}  % Coloured text etc.
\definecolor{orioles}{HTML}{F6511D}
\definecolor{ucla}{HTML}{FFB400}
\definecolor{vivid}{HTML}{00A6ED}
\definecolor{apple}{HTML}{7FB800}
\definecolor{pruss}{HTML}{0D2C54}

% ****************************************************************************************************
% 1. Configure classicthesis for your needs here, e.g., remove "drafting" below
% in order to deactivate the time-stamp on the pages
% ****************************************************************************************************
\PassOptionsToPackage{
  drafting=false,    % print version information on the bottom of the pages
  tocaligned=false, % the left column of the toc will be aligned (no indentation)
  dottedtoc=true,  % page numbers in ToC flushed right
  eulerchapternumbers=true, % use AMS Euler for chapter font (otherwise Palatino)
  linedheaders=false,       % chaper headers will have line above and beneath
  floatperchapter=true,     % numbering per chapter for all floats (i.e., Figure 1.1)
  eulermath=true,  % use awesome Euler fonts for mathematical formulae (only with pdfLaTeX)
  beramono=true,    % toggle a nice monospaced font (w/ bold)
  palatino=false,    % deactivate standard font for loading another one, see the last section at the end of this file for suggestions
  style=classicthesis % classicthesis, arsclassica
}{classicthesis}

% ****************************************************************************************************
% 2. Personal data and user ad-hoc commands
% ****************************************************************************************************
\newcommand{\Title}{Abstractions via Program Transformations}

\newcommand{\Subtitle}{phd thesis proposal\xspace}
\newcommand{\Logo}{img/fi-logo}
\newcommand{\Author}{Henrich Lauko\xspace}
\newcommand{\Supervisor}{prof. RNDr. Jiří Barnat, Ph.D.\xspace}
\newcommand{\Consultant}{RNDr. Petr Ročkai, Ph.D.\xspace}
\newcommand{\Department}{Faculty of Informatics\xspace}
\newcommand{\University}{Masaryk University\xspace}
\newcommand{\Year}{September 2019\xspace}
\newcommand{\myVersion}{version 0.1\xspace}

% ********************************************************************
% Setup, finetuning, and useful commands
% ********************************************************************
\newcounter{dummy} % necessary for correct hyperlinks (to index, bib, etc.)
\newlength{\abcd} % for ab..z string length calculation
\providecommand{\mLyX}{L\kern-.1667em\lower.25em\hbox{Y}\kern-.125emX\@}
\newcommand{\ie}{i.\,e.}
\newcommand{\Ie}{I.\,e.}
\newcommand{\eg}{e.\,g.}
\newcommand{\Eg}{E.\,g.}

% ****************************************************************************************************
% 3. Loading some handy packages
% ****************************************************************************************************
% ********************************************************************
% Packages with options that might require adjustments
% ********************************************************************
%\PassOptionsToPackage{ngerman,american}{babel}   % change this to your language(s)
% Spanish languages need extra options in order to work with this template
%\PassOptionsToPackage{spanish,es-lcroman}{babel}
\usepackage{babel}
\usepackage{adjustbox}

\usepackage{enumitem}

\usepackage{minted}
%\usepackage{inconsolata}
\setminted{
    frame=lines,
    rulecolor=pruss,
    framesep=1mm,
    autogobble,
    numbersep=8pt,
    baselinestretch=1.2,
    fontsize=\footnotesize,
    escapeinside=||,
    mathescape=true
}
\usemintedstyle{lovelace}

\newmintinline[llvmint]{llvm}{}

\newcommand{\code}[1]{\texttt{#1}}
\newcommand{\ccode}[1]{\mintinline{cpp}{#1}}
\newcommand{\lcode}[1]{\mintinline{llvm}{#1}}

\definecolor{varcolor}{HTML}{B83838}
\newcommand{\var}[1]{\hspace{-0.5em}\color{varcolor}\text{#1}}

\usepackage{csquotes}

\PassOptionsToPackage{%
    %backend=biber, %instead of bibtex
	backend=bibtex8,bibencoding=ascii,%
	language=auto,%
	style=alphabetic,%
        citestyle=alphabetic,
        firstinits=true,
    %style=authoryear-comp, % Author 1999, 2010
    %bibstyle=authoryear,dashed=false, % dashed: substitute rep. author with ---
    sorting=nyt, % name, year, title
    maxbibnames=10, % default: 3, et al.
    %backref=true,%
    natbib=true % natbib compatibility mode (\citep and \citet still work)
}{biblatex}
\usepackage{biblatex}

\renewcommand\mkbibnamefamily[1]{\textsc{#1}}

\newcommand{\prule}{
    {\color{pruss} \hrule}%
}

\PassOptionsToPackage{fleqn}{amsmath} % math environments and more by the AMS
\usepackage{amsmath,amsfonts,amssymb,amsthm}

\usepackage{framed,  etoolbox}
  \colorlet{framecolor}{pruss!40}
\usepackage{thmtools}

\makeatletter
\define@key{thmdef}{frame}[{}]{%
 \thmt@trytwice{}{%
 \RequirePackage{framed}%
 \RequirePackage{thm-patch}%
 \addtotheorempreheadhook[\thmt@envname]{\medskip \par}%
 \addtotheorempostfoothook[\thmt@envname]{\medskip}%
 }%
}
\makeatother

\declaretheorem[numberwithin=section, frame]{definition}
\declaretheorem[numberwithin=section, frame]{example}

\DeclareSymbolFont{operators}{\encodingdefault}{ppl}{m}{n}
\DeclareMathAlphabet{\mathbf}{\encodingdefault}{ppl}{bx}{n}
\DeclareMathAlphabet{\mathit}{\encodingdefault}{ppl}{m}{it}

% ********************************************************************
% General useful packages
% ********************************************************************
\PassOptionsToPackage{T1}{fontenc} % T2A for cyrillics

%\usepackage{fontspec}
\usepackage{textcomp} % fix warning with missing font shapes
\usepackage{scrhack} % fix warnings when using KOMA with listings package
\usepackage{xspace} % to get the spacing after macros right
\usepackage{mparhack} % get marginpar right
\usepackage{marginnote}
\PassOptionsToPackage{printonlyused,smaller}{acronym}
  \usepackage{acronym} % nice macros for handling all acronyms in the thesis
  %\renewcommand{\bflabel}[1]{{#1}\hfill} % fix the list of acronyms --> no longer working
  %\renewcommand*{\acsfont}[1]{\textsc{#1}}
  %\renewcommand*{\aclabelfont}[1]{\acsfont{#1}}
  %\def\bflabel#1{{#1\hfill}}
  \def\bflabel#1{{\acsfont{#1}\hfill}}
  \def\aclabelfont#1{\acsfont{#1}}

\renewcommand*{\aclabelfont}[1]{\acsfont{#1}}

\usepackage{microtype}

\clubpenalty  = 8000        % help suppress orphans, default = 4,000 (?), from 0 to 10 000 (from 300 to 1 000 recommended, 10 000 not recommended)
\widowpenalty = 8000        % help suppress widows,  default = 4,000 (?), from 0 to 10 000 (from 300 to 1 000 recommended, 10 000 not recommended)
\SetTracking{encoding={*}, shape=sc}{40}

\usepackage{syntax}

% ****************************************************************************************************
% 4. Setup floats: tables, (sub)figures, and captions
% ****************************************************************************************************
\usepackage[strict]{changepage}%
\usepackage{colortbl}
\usepackage{booktabs,tabularx}%
    \setlength{\extrarowheight}{3pt} % increase table row height
\newcommand{\tableheadline}[1]{\multicolumn{1}{c}{\spacedlowsmallcaps{#1}}}
\newcommand{\myfloatalign}{\centering} % to be used with each float for alignment
\usepackage{caption}
\captionsetup{format=plain,font=small, textfont=it, labelfont=bf}
\usepackage{subfig}

\newcolumntype{g}{>{\columncolor{pruss!10}}c}

\usepackage{xkeyval}
\usepackage{subfig}
\usepackage{graphicx}
\usepackage{calc}

% ****************************************************************************************************
% 6. PDFLaTeX, hyperreferences and citation backreferences
% ****************************************************************************************************
\PassOptionsToPackage{xetex,hyperfootnotes=false,pdfpagelabels}{hyperref}
\usepackage{hyperref}  % backref linktocpage pagebackref
%\pdfcompresslevel=9
%\pdfadjustspacing=1
\PassOptionsToPackage{xetex}{graphicx}
\usepackage{graphicx}
\usepackage{pdfpages}

% ********************************************************************
% Hyperreferences
% ********************************************************************
\hypersetup{%
    %draft, % = no hyperlinking at all (useful in b/w printouts)
    colorlinks=true, linktocpage=true, pdfstartpage=3, pdfstartview=FitV,%
    % uncomment the following line if you want to have black links (e.g., for printing)
    %colorlinks=false, linktocpage=false, pdfstartpage=3, pdfstartview=FitV, pdfborder={0 0 0},%
    breaklinks=true, pdfpagemode=UseNone, pageanchor=true, pdfpagemode=UseOutlines,%
    plainpages=false, bookmarksnumbered, bookmarksopen=true, bookmarksopenlevel=1,%
    hypertexnames=true, pdfhighlight=/O,%nesting=true,%frenchlinks,%
    urlcolor=webbrown, linkcolor=pruss, citecolor=pruss, %pagecolor=RoyalBlue,%
    %urlcolor=Black, linkcolor=Black, citecolor=Black, %pagecolor=Black,%
    pdftitle={\Title},%
    pdfauthor={\textcopyright\ \Author, \University, \Department},%
    pdfsubject={},%
    pdfkeywords={},%
    pdfcreator={pdfLaTeX},%
    pdfproducer={LaTeX with hyperref and classicthesis}%
}

% ********************************************************************
% Setup autoreferences
% ********************************************************************
\makeatletter
\@ifpackageloaded{babel}%
    {%
       \addto\extrasamerican{%
			\renewcommand*{\figureautorefname}{Figure}%
			\renewcommand*{\tableautorefname}{Table}%
			\renewcommand*{\partautorefname}{Part}%
			\renewcommand*{\chapterautorefname}{Chapter}%
			\renewcommand*{\sectionautorefname}{Section}%
			\renewcommand*{\subsectionautorefname}{Section}%
			\renewcommand*{\subsubsectionautorefname}{Section}%
                }%
       \addto\extrasngerman{%
			\renewcommand*{\paragraphautorefname}{Absatz}%
			\renewcommand*{\subparagraphautorefname}{Unterabsatz}%
			\renewcommand*{\footnoteautorefname}{Fu\"snote}%
			\renewcommand*{\FancyVerbLineautorefname}{Zeile}%
			\renewcommand*{\theoremautorefname}{Theorem}%
			\renewcommand*{\appendixautorefname}{Anhang}%
			\renewcommand*{\equationautorefname}{Gleichung}%
			\renewcommand*{\itemautorefname}{Punkt}%
                }%
            % Fix to getting autorefs for subfigures right (thanks to Belinda Vogt for changing the definition)
            \providecommand{\subfigureautorefname}{\figureautorefname}%
    }{\relax}
\makeatother


% ********************************************************************
% Development Stuff
% ********************************************************************
\listfiles

% ********************************************************************
% Last, but not least...
% ********************************************************************
\usepackage{classicthesis}

\areaset[current]{312pt}{657pt}
\setlength{\marginparwidth}{9.5em}%
\setlength{\marginparsep}{2em}%

\usepackage{xargs}
\usepackage[colorinlistoftodos,prependcaption,textsize=tiny]{todonotes}
\newcommandx{\add}[2][1=]{\todo[inline, linecolor=red,backgroundcolor=red!25,bordercolor=red,#1]{#2}}
\newcommandx{\change}[2][1=]{\todo[inline,linecolor=blue,backgroundcolor=blue!25,bordercolor=blue,#1]{#2}}
\newcommandx{\info}[2][1=]{\todo[inline,linecolor=OliveGreen,backgroundcolor=OliveGreen!25,bordercolor=OliveGreen,#1]{#2}}
\newcommandx{\improvement}[2][1=]{\todo[inline,linecolor=Plum,backgroundcolor=Plum!25,bordercolor=Plum,#1]{#2}}
\newcommandx{\thiswillnotshow}[2][1=]{\todo[disable,#1]{#2}}

\makeatletter
\newcommand*{\textoverline}[1]{$\overline{\hbox{#1}}\m@th$}
\makeatother

\usepackage{tikz-picture}

% ****************************************************************************************************
% 8. Further adjustments (experimental)
% ****************************************************************************************************
% ********************************************************************
% Changing the text area
% ********************************************************************
\linespread{1.05} % a bit more for Palatino
\areaset[current]{312pt}{720pt} % 686 (factor 2.2) + 33 head + 42 head \the\footskip
\setlength{\marginparwidth}{9em}%
\setlength{\marginparsep}{2em}%

% ********************************************************************
% Using different fonts
% ********************************************************************
%\setmathfont{MnSymbol}
%\setmathfont[range=\mathup/{num,latin,Latin,greek,Greek}]{Minion Pro}
%\setmathfont[range=\mathbfup/{num,latin,Latin,greek,Greek}]{MinionPro-Bold}
%\setmathfont[range=\mathit/{num,latin,Latin,greek,Greek}]{MinionPro-It}
%\setmathfont[range=\mathbfit/{num,latin,Latin,greek,Greek}]{MinionPro-BoldIt}
%\setmathrm{Minion Pro}

%\usepackage{unicode-math}

%\setmainfont[Numbers={OldStyle,Proportional},SmallCapsFeatures={LetterSpace=6}]{Minion Pro}

%\usepackage[oldstylenums]{kpfonts} % oldstyle notextcomp
%\usepackage[osf]{libertine}
%\usepackage[light,condensed,math]{iwona}
%\renewcommand{\sfdefault}{iwona}
%\usepackage{lmodern} % <-- no osf support :-(
%\usepackage{cfr-lm} %
%\usepackage[urw-garamond]{mathdesign} %<-- no osf support :-(
%\usepackage[default,osfigures]{opensans} % scale=0.95
%\usepackage[sfdefault]{FiraSans}

\usepackage{pifont}

%\usepackage[math-style=upright]{unicode-math}
\usepackage{unicode-math}

\setmainfont[Numbers={OldStyle,Proportional},SmallCapsFeatures={LetterSpace=6}]{Minion Pro}
\setmathfont[Scale=MatchLowercase]{STIX2Math.otf}
\setmonofont[Scale=MatchLowercase]{Inconsolata}

\usepackage{polyglossia}
\setmainlanguage{english}
\setotherlanguage{czech}

\usepackage{tocloft}
\renewcommand{\cftchapfont}{\normalfont\scshape}
\renewcommand{\cftpartfont}{\color{CTtitle}\normalfont\scshape}%
\setlength{\cftbeforechapskip}{7pt}
%\renewcommand\cftsecafterpnum{\vskip15pt}

% ****************************************************************************************************

\usepackage{sidenotes}

\newenvironment{summary}
{\bigskip\hrule\bigskip\itshape}
{\bigskip\hrule}

% ****************************************************************************************************
% If you like the classicthesis, then I would appreciate a postcard.
% My address can be found in the file ClassicThesis.pdf. A collection
% of the postcards I received so far is available online at
% http://postcards.miede.de
% ****************************************************************************************************
\PassOptionsToPackage{utf8}{inputenc}
\usepackage{inputenc}

\usepackage[dvipsnames]{xcolor}  % Coloured text etc.
\definecolor{orioles}{HTML}{F6511D}
\definecolor{ucla}{HTML}{FFB400}
\definecolor{vivid}{HTML}{00A6ED}
\definecolor{apple}{HTML}{7FB800}
\definecolor{pruss}{HTML}{0D2C54}

% ****************************************************************************************************
% 1. Configure classicthesis for your needs here, e.g., remove "drafting" below
% in order to deactivate the time-stamp on the pages
% ****************************************************************************************************
\PassOptionsToPackage{
  drafting=false,    % print version information on the bottom of the pages
  tocaligned=false, % the left column of the toc will be aligned (no indentation)
  dottedtoc=true,  % page numbers in ToC flushed right
  eulerchapternumbers=true, % use AMS Euler for chapter font (otherwise Palatino)
  linedheaders=false,       % chaper headers will have line above and beneath
  floatperchapter=true,     % numbering per chapter for all floats (i.e., Figure 1.1)
  eulermath=true,  % use awesome Euler fonts for mathematical formulae (only with pdfLaTeX)
  beramono=true,    % toggle a nice monospaced font (w/ bold)
  palatino=false,    % deactivate standard font for loading another one, see the last section at the end of this file for suggestions
  style=classicthesis % classicthesis, arsclassica
}{classicthesis}

% ****************************************************************************************************
% 2. Personal data and user ad-hoc commands
% ****************************************************************************************************
\newcommand{\Title}{Abstractions via Program Transformations}

\newcommand{\Subtitle}{phd thesis proposal\xspace}
\newcommand{\Logo}{img/fi-logo}
\newcommand{\Author}{Henrich Lauko\xspace}
\newcommand{\Supervisor}{prof. RNDr. Jiří Barnat, Ph.D.\xspace}
\newcommand{\Consultant}{RNDr. Petr Ročkai, Ph.D.\xspace}
\newcommand{\Department}{Faculty of Informatics\xspace}
\newcommand{\University}{Masaryk University\xspace}
\newcommand{\Year}{September 2019\xspace}
\newcommand{\myVersion}{version 0.1\xspace}

% ********************************************************************
% Setup, finetuning, and useful commands
% ********************************************************************
\newcounter{dummy} % necessary for correct hyperlinks (to index, bib, etc.)
\newlength{\abcd} % for ab..z string length calculation
\providecommand{\mLyX}{L\kern-.1667em\lower.25em\hbox{Y}\kern-.125emX\@}
\newcommand{\ie}{i.\,e.}
\newcommand{\Ie}{I.\,e.}
\newcommand{\eg}{e.\,g.}
\newcommand{\Eg}{E.\,g.}

% ****************************************************************************************************
% 3. Loading some handy packages
% ****************************************************************************************************
% ********************************************************************
% Packages with options that might require adjustments
% ********************************************************************
%\PassOptionsToPackage{ngerman,american}{babel}   % change this to your language(s)
% Spanish languages need extra options in order to work with this template
%\PassOptionsToPackage{spanish,es-lcroman}{babel}
\usepackage{babel}
\usepackage{adjustbox}

\usepackage{enumitem}

\usepackage{minted}
%\usepackage{inconsolata}
\setminted{
    frame=lines,
    rulecolor=pruss,
    framesep=1mm,
    autogobble,
    numbersep=8pt,
    baselinestretch=1.2,
    fontsize=\footnotesize,
    escapeinside=||,
    mathescape=true
}
\usemintedstyle{lovelace}

\newmintinline[llvmint]{llvm}{}

\newcommand{\code}[1]{\texttt{#1}}
\newcommand{\ccode}[1]{\mintinline{cpp}{#1}}
\newcommand{\lcode}[1]{\mintinline{llvm}{#1}}

\definecolor{varcolor}{HTML}{B83838}
\newcommand{\var}[1]{\hspace{-0.5em}\color{varcolor}\text{#1}}

\usepackage{csquotes}

\PassOptionsToPackage{%
    %backend=biber, %instead of bibtex
	backend=bibtex8,bibencoding=ascii,%
	language=auto,%
	style=alphabetic,%
        citestyle=alphabetic,
        firstinits=true,
    %style=authoryear-comp, % Author 1999, 2010
    %bibstyle=authoryear,dashed=false, % dashed: substitute rep. author with ---
    sorting=nyt, % name, year, title
    maxbibnames=10, % default: 3, et al.
    %backref=true,%
    natbib=true % natbib compatibility mode (\citep and \citet still work)
}{biblatex}
\usepackage{biblatex}

\renewcommand\mkbibnamefamily[1]{\textsc{#1}}

\newcommand{\prule}{
    {\color{pruss} \hrule}%
}

\PassOptionsToPackage{fleqn}{amsmath} % math environments and more by the AMS
\usepackage{amsmath,amsfonts,amssymb,amsthm}

\usepackage{framed,  etoolbox}
  \colorlet{framecolor}{pruss!40}
\usepackage{thmtools}

\makeatletter
\define@key{thmdef}{frame}[{}]{%
 \thmt@trytwice{}{%
 \RequirePackage{framed}%
 \RequirePackage{thm-patch}%
 \addtotheorempreheadhook[\thmt@envname]{\medskip \par}%
 \addtotheorempostfoothook[\thmt@envname]{\medskip}%
 }%
}
\makeatother

\declaretheorem[numberwithin=section, frame]{definition}
\declaretheorem[numberwithin=section, frame]{example}

\DeclareSymbolFont{operators}{\encodingdefault}{ppl}{m}{n}
\DeclareMathAlphabet{\mathbf}{\encodingdefault}{ppl}{bx}{n}
\DeclareMathAlphabet{\mathit}{\encodingdefault}{ppl}{m}{it}

% ********************************************************************
% General useful packages
% ********************************************************************
\PassOptionsToPackage{T1}{fontenc} % T2A for cyrillics

%\usepackage{fontspec}
\usepackage{textcomp} % fix warning with missing font shapes
\usepackage{scrhack} % fix warnings when using KOMA with listings package
\usepackage{xspace} % to get the spacing after macros right
\usepackage{mparhack} % get marginpar right
\usepackage{marginnote}
\PassOptionsToPackage{printonlyused,smaller}{acronym}
  \usepackage{acronym} % nice macros for handling all acronyms in the thesis
  %\renewcommand{\bflabel}[1]{{#1}\hfill} % fix the list of acronyms --> no longer working
  %\renewcommand*{\acsfont}[1]{\textsc{#1}}
  %\renewcommand*{\aclabelfont}[1]{\acsfont{#1}}
  %\def\bflabel#1{{#1\hfill}}
  \def\bflabel#1{{\acsfont{#1}\hfill}}
  \def\aclabelfont#1{\acsfont{#1}}

\renewcommand*{\aclabelfont}[1]{\acsfont{#1}}

\usepackage{microtype}

\clubpenalty  = 8000        % help suppress orphans, default = 4,000 (?), from 0 to 10 000 (from 300 to 1 000 recommended, 10 000 not recommended)
\widowpenalty = 8000        % help suppress widows,  default = 4,000 (?), from 0 to 10 000 (from 300 to 1 000 recommended, 10 000 not recommended)
\SetTracking{encoding={*}, shape=sc}{40}

\usepackage{syntax}

% ****************************************************************************************************
% 4. Setup floats: tables, (sub)figures, and captions
% ****************************************************************************************************
\usepackage[strict]{changepage}%
\usepackage{colortbl}
\usepackage{booktabs,tabularx}%
    \setlength{\extrarowheight}{3pt} % increase table row height
\newcommand{\tableheadline}[1]{\multicolumn{1}{c}{\spacedlowsmallcaps{#1}}}
\newcommand{\myfloatalign}{\centering} % to be used with each float for alignment
\usepackage{caption}
\captionsetup{format=plain,font=small, textfont=it, labelfont=bf}
\usepackage{subfig}

\newcolumntype{g}{>{\columncolor{pruss!10}}c}

\usepackage{xkeyval}
\usepackage{subfig}
\usepackage{graphicx}
\usepackage{calc}

% ****************************************************************************************************
% 6. PDFLaTeX, hyperreferences and citation backreferences
% ****************************************************************************************************
\PassOptionsToPackage{xetex,hyperfootnotes=false,pdfpagelabels}{hyperref}
\usepackage{hyperref}  % backref linktocpage pagebackref
%\pdfcompresslevel=9
%\pdfadjustspacing=1
\PassOptionsToPackage{xetex}{graphicx}
\usepackage{graphicx}
\usepackage{pdfpages}

% ********************************************************************
% Hyperreferences
% ********************************************************************
\hypersetup{%
    %draft, % = no hyperlinking at all (useful in b/w printouts)
    colorlinks=true, linktocpage=true, pdfstartpage=3, pdfstartview=FitV,%
    % uncomment the following line if you want to have black links (e.g., for printing)
    %colorlinks=false, linktocpage=false, pdfstartpage=3, pdfstartview=FitV, pdfborder={0 0 0},%
    breaklinks=true, pdfpagemode=UseNone, pageanchor=true, pdfpagemode=UseOutlines,%
    plainpages=false, bookmarksnumbered, bookmarksopen=true, bookmarksopenlevel=1,%
    hypertexnames=true, pdfhighlight=/O,%nesting=true,%frenchlinks,%
    urlcolor=webbrown, linkcolor=pruss, citecolor=pruss, %pagecolor=RoyalBlue,%
    %urlcolor=Black, linkcolor=Black, citecolor=Black, %pagecolor=Black,%
    pdftitle={\Title},%
    pdfauthor={\textcopyright\ \Author, \University, \Department},%
    pdfsubject={},%
    pdfkeywords={},%
    pdfcreator={pdfLaTeX},%
    pdfproducer={LaTeX with hyperref and classicthesis}%
}

% ********************************************************************
% Setup autoreferences
% ********************************************************************
\makeatletter
\@ifpackageloaded{babel}%
    {%
       \addto\extrasamerican{%
			\renewcommand*{\figureautorefname}{Figure}%
			\renewcommand*{\tableautorefname}{Table}%
			\renewcommand*{\partautorefname}{Part}%
			\renewcommand*{\chapterautorefname}{Chapter}%
			\renewcommand*{\sectionautorefname}{Section}%
			\renewcommand*{\subsectionautorefname}{Section}%
			\renewcommand*{\subsubsectionautorefname}{Section}%
                }%
       \addto\extrasngerman{%
			\renewcommand*{\paragraphautorefname}{Absatz}%
			\renewcommand*{\subparagraphautorefname}{Unterabsatz}%
			\renewcommand*{\footnoteautorefname}{Fu\"snote}%
			\renewcommand*{\FancyVerbLineautorefname}{Zeile}%
			\renewcommand*{\theoremautorefname}{Theorem}%
			\renewcommand*{\appendixautorefname}{Anhang}%
			\renewcommand*{\equationautorefname}{Gleichung}%
			\renewcommand*{\itemautorefname}{Punkt}%
                }%
            % Fix to getting autorefs for subfigures right (thanks to Belinda Vogt for changing the definition)
            \providecommand{\subfigureautorefname}{\figureautorefname}%
    }{\relax}
\makeatother


% ********************************************************************
% Development Stuff
% ********************************************************************
\listfiles

% ********************************************************************
% Last, but not least...
% ********************************************************************
\usepackage{classicthesis}

\areaset[current]{312pt}{657pt}
\setlength{\marginparwidth}{9.5em}%
\setlength{\marginparsep}{2em}%

\usepackage{xargs}
\usepackage[colorinlistoftodos,prependcaption,textsize=tiny]{todonotes}
\newcommandx{\add}[2][1=]{\todo[inline, linecolor=red,backgroundcolor=red!25,bordercolor=red,#1]{#2}}
\newcommandx{\change}[2][1=]{\todo[inline,linecolor=blue,backgroundcolor=blue!25,bordercolor=blue,#1]{#2}}
\newcommandx{\info}[2][1=]{\todo[inline,linecolor=OliveGreen,backgroundcolor=OliveGreen!25,bordercolor=OliveGreen,#1]{#2}}
\newcommandx{\improvement}[2][1=]{\todo[inline,linecolor=Plum,backgroundcolor=Plum!25,bordercolor=Plum,#1]{#2}}
\newcommandx{\thiswillnotshow}[2][1=]{\todo[disable,#1]{#2}}

\makeatletter
\newcommand*{\textoverline}[1]{$\overline{\hbox{#1}}\m@th$}
\makeatother

\usepackage{tikz-picture}

% ****************************************************************************************************
% 8. Further adjustments (experimental)
% ****************************************************************************************************
% ********************************************************************
% Changing the text area
% ********************************************************************
\linespread{1.05} % a bit more for Palatino
\areaset[current]{312pt}{720pt} % 686 (factor 2.2) + 33 head + 42 head \the\footskip
\setlength{\marginparwidth}{9em}%
\setlength{\marginparsep}{2em}%

% ********************************************************************
% Using different fonts
% ********************************************************************
%\setmathfont{MnSymbol}
%\setmathfont[range=\mathup/{num,latin,Latin,greek,Greek}]{Minion Pro}
%\setmathfont[range=\mathbfup/{num,latin,Latin,greek,Greek}]{MinionPro-Bold}
%\setmathfont[range=\mathit/{num,latin,Latin,greek,Greek}]{MinionPro-It}
%\setmathfont[range=\mathbfit/{num,latin,Latin,greek,Greek}]{MinionPro-BoldIt}
%\setmathrm{Minion Pro}

%\usepackage{unicode-math}

%\setmainfont[Numbers={OldStyle,Proportional},SmallCapsFeatures={LetterSpace=6}]{Minion Pro}

%\usepackage[oldstylenums]{kpfonts} % oldstyle notextcomp
%\usepackage[osf]{libertine}
%\usepackage[light,condensed,math]{iwona}
%\renewcommand{\sfdefault}{iwona}
%\usepackage{lmodern} % <-- no osf support :-(
%\usepackage{cfr-lm} %
%\usepackage[urw-garamond]{mathdesign} %<-- no osf support :-(
%\usepackage[default,osfigures]{opensans} % scale=0.95
%\usepackage[sfdefault]{FiraSans}

\usepackage{pifont}

%\usepackage[math-style=upright]{unicode-math}
\usepackage{unicode-math}

\setmainfont[Numbers={OldStyle,Proportional},SmallCapsFeatures={LetterSpace=6}]{Minion Pro}
\setmathfont[Scale=MatchLowercase]{STIX2Math.otf}
\setmonofont[Scale=MatchLowercase]{Inconsolata}

\usepackage{polyglossia}
\setmainlanguage{english}
\setotherlanguage{czech}

\usepackage{tocloft}
\renewcommand{\cftchapfont}{\normalfont\scshape}
\renewcommand{\cftpartfont}{\color{CTtitle}\normalfont\scshape}%
\setlength{\cftbeforechapskip}{7pt}
%\renewcommand\cftsecafterpnum{\vskip15pt}

% ****************************************************************************************************

\usepackage{sidenotes}

\newenvironment{summary}
{\bigskip\hrule\bigskip\itshape}
{\bigskip\hrule}

% ****************************************************************************************************
% If you like the classicthesis, then I would appreciate a postcard.
% My address can be found in the file ClassicThesis.pdf. A collection
% of the postcards I received so far is available online at
% http://postcards.miede.de
% ****************************************************************************************************
\PassOptionsToPackage{utf8}{inputenc}
\usepackage{inputenc}

\usepackage[dvipsnames]{xcolor}  % Coloured text etc.
\definecolor{orioles}{HTML}{F6511D}
\definecolor{ucla}{HTML}{FFB400}
\definecolor{vivid}{HTML}{00A6ED}
\definecolor{apple}{HTML}{7FB800}
\definecolor{pruss}{HTML}{0D2C54}

% ****************************************************************************************************
% 1. Configure classicthesis for your needs here, e.g., remove "drafting" below
% in order to deactivate the time-stamp on the pages
% ****************************************************************************************************
\PassOptionsToPackage{
  drafting=false,    % print version information on the bottom of the pages
  tocaligned=false, % the left column of the toc will be aligned (no indentation)
  dottedtoc=true,  % page numbers in ToC flushed right
  eulerchapternumbers=true, % use AMS Euler for chapter font (otherwise Palatino)
  linedheaders=false,       % chaper headers will have line above and beneath
  floatperchapter=true,     % numbering per chapter for all floats (i.e., Figure 1.1)
  eulermath=true,  % use awesome Euler fonts for mathematical formulae (only with pdfLaTeX)
  beramono=true,    % toggle a nice monospaced font (w/ bold)
  palatino=false,    % deactivate standard font for loading another one, see the last section at the end of this file for suggestions
  style=classicthesis % classicthesis, arsclassica
}{classicthesis}

% ****************************************************************************************************
% 2. Personal data and user ad-hoc commands
% ****************************************************************************************************
\newcommand{\Title}{Abstractions via Program Transformations}

\newcommand{\Subtitle}{phd thesis proposal\xspace}
\newcommand{\Logo}{img/fi-logo}
\newcommand{\Author}{Henrich Lauko\xspace}
\newcommand{\Supervisor}{prof. RNDr. Jiří Barnat, Ph.D.\xspace}
\newcommand{\Consultant}{RNDr. Petr Ročkai, Ph.D.\xspace}
\newcommand{\Department}{Faculty of Informatics\xspace}
\newcommand{\University}{Masaryk University\xspace}
\newcommand{\Year}{September 2019\xspace}
\newcommand{\myVersion}{version 0.1\xspace}

% ********************************************************************
% Setup, finetuning, and useful commands
% ********************************************************************
\newcounter{dummy} % necessary for correct hyperlinks (to index, bib, etc.)
\newlength{\abcd} % for ab..z string length calculation
\providecommand{\mLyX}{L\kern-.1667em\lower.25em\hbox{Y}\kern-.125emX\@}
\newcommand{\ie}{i.\,e.}
\newcommand{\Ie}{I.\,e.}
\newcommand{\eg}{e.\,g.}
\newcommand{\Eg}{E.\,g.}

% ****************************************************************************************************
% 3. Loading some handy packages
% ****************************************************************************************************
% ********************************************************************
% Packages with options that might require adjustments
% ********************************************************************
%\PassOptionsToPackage{ngerman,american}{babel}   % change this to your language(s)
% Spanish languages need extra options in order to work with this template
%\PassOptionsToPackage{spanish,es-lcroman}{babel}
\usepackage{babel}
\usepackage{adjustbox}

\usepackage{enumitem}

\usepackage{minted}
%\usepackage{inconsolata}
\setminted{
    frame=lines,
    rulecolor=pruss,
    framesep=1mm,
    autogobble,
    numbersep=8pt,
    baselinestretch=1.2,
    fontsize=\footnotesize,
    escapeinside=||,
    mathescape=true
}
\usemintedstyle{lovelace}

\newmintinline[llvmint]{llvm}{}

\newcommand{\code}[1]{\texttt{#1}}
\newcommand{\ccode}[1]{\mintinline{cpp}{#1}}
\newcommand{\lcode}[1]{\mintinline{llvm}{#1}}

\definecolor{varcolor}{HTML}{B83838}
\newcommand{\var}[1]{\hspace{-0.5em}\color{varcolor}\text{#1}}
\newcommand{\op}[1]{\hspace{-0.5em}\color{varcolor}\text{#1}}

\usepackage{csquotes}

\PassOptionsToPackage{%
    %backend=biber, %instead of bibtex
	backend=bibtex8,bibencoding=ascii,%
	language=auto,%
	style=alphabetic,%
        citestyle=alphabetic,
        firstinits=true,
    %style=authoryear-comp, % Author 1999, 2010
    %bibstyle=authoryear,dashed=false, % dashed: substitute rep. author with ---
    sorting=nyt, % name, year, title
    maxbibnames=10, % default: 3, et al.
    %backref=true,%
    natbib=true % natbib compatibility mode (\citep and \citet still work)
}{biblatex}
\usepackage{biblatex}

\renewcommand\mkbibnamefamily[1]{\textsc{#1}}

\newcommand{\prule}{
    {\color{pruss} \hrule}%
}

\PassOptionsToPackage{fleqn}{amsmath} % math environments and more by the AMS
\usepackage{amsmath,amsfonts,amssymb,amsthm}

\usepackage{framed,  etoolbox}
  \colorlet{framecolor}{pruss!40}
\usepackage{thmtools}

\makeatletter
\define@key{thmdef}{frame}[{}]{%
 \thmt@trytwice{}{%
 \RequirePackage{framed}%
 \RequirePackage{thm-patch}%
 \addtotheorempreheadhook[\thmt@envname]{\medskip \par}%
 \addtotheorempostfoothook[\thmt@envname]{\medskip}%
 }%
}
\makeatother

\declaretheorem[numberwithin=section, frame]{definition}
\declaretheorem[numberwithin=section, frame]{example}

\DeclareSymbolFont{operators}{\encodingdefault}{ppl}{m}{n}
\DeclareMathAlphabet{\mathbf}{\encodingdefault}{ppl}{bx}{n}
\DeclareMathAlphabet{\mathit}{\encodingdefault}{ppl}{m}{it}

% ********************************************************************
% General useful packages
% ********************************************************************
\PassOptionsToPackage{T1}{fontenc} % T2A for cyrillics

\usepackage{fontspec}
\usepackage{textcomp} % fix warning with missing font shapes
\usepackage{scrhack} % fix warnings when using KOMA with listings package
\usepackage{xspace} % to get the spacing after macros right
\usepackage{mparhack} % get marginpar right
\usepackage{marginnote}
\PassOptionsToPackage{printonlyused,smaller}{acronym}
  \usepackage{acronym} % nice macros for handling all acronyms in the thesis
  %\renewcommand{\bflabel}[1]{{#1}\hfill} % fix the list of acronyms --> no longer working
  %\renewcommand*{\acsfont}[1]{\textsc{#1}}
  %\renewcommand*{\aclabelfont}[1]{\acsfont{#1}}
  %\def\bflabel#1{{#1\hfill}}
  \def\bflabel#1{{\acsfont{#1}\hfill}}
  \def\aclabelfont#1{\acsfont{#1}}

\renewcommand*{\aclabelfont}[1]{\acsfont{#1}}

\usepackage{microtype}

\clubpenalty  = 8000        % help suppress orphans, default = 4,000 (?), from 0 to 10 000 (from 300 to 1 000 recommended, 10 000 not recommended)
\widowpenalty = 8000        % help suppress widows,  default = 4,000 (?), from 0 to 10 000 (from 300 to 1 000 recommended, 10 000 not recommended)
\SetTracking{encoding={*}, shape=sc}{40}

\usepackage{syntax}

% ****************************************************************************************************
% 4. Setup floats: tables, (sub)figures, and captions
% ****************************************************************************************************
\usepackage[strict]{changepage}%
\usepackage{colortbl}
\usepackage{booktabs,tabularx}%
    \setlength{\extrarowheight}{3pt} % increase table row height
\newcommand{\tableheadline}[1]{\multicolumn{1}{c}{\spacedlowsmallcaps{#1}}}
\newcommand{\myfloatalign}{\centering} % to be used with each float for alignment
\usepackage{caption}
\captionsetup{format=plain,font=small, textfont=it, labelfont=bf}
\usepackage{subfig}

\newcolumntype{g}{>{\columncolor{pruss!10}}c}

\usepackage{xkeyval}
\usepackage{subfig}
\usepackage{graphicx}
\usepackage{calc}

% ****************************************************************************************************
% 6. PDFLaTeX, hyperreferences and citation backreferences
% ****************************************************************************************************
\PassOptionsToPackage{xetex,hyperfootnotes=false,pdfpagelabels}{hyperref}
\usepackage{hyperref}  % backref linktocpage pagebackref
%\pdfcompresslevel=9
%\pdfadjustspacing=1
\PassOptionsToPackage{xetex}{graphicx}
\usepackage{graphicx}
\usepackage{pdfpages}

% ********************************************************************
% Hyperreferences
% ********************************************************************
\hypersetup{%
    %draft, % = no hyperlinking at all (useful in b/w printouts)
    colorlinks=true, linktocpage=true, pdfstartpage=3, pdfstartview=FitV,%
    % uncomment the following line if you want to have black links (e.g., for printing)
    %colorlinks=false, linktocpage=false, pdfstartpage=3, pdfstartview=FitV, pdfborder={0 0 0},%
    breaklinks=true, pdfpagemode=UseNone, pageanchor=true, pdfpagemode=UseOutlines,%
    plainpages=false, bookmarksnumbered, bookmarksopen=true, bookmarksopenlevel=1,%
    hypertexnames=true, pdfhighlight=/O,%nesting=true,%frenchlinks,%
    urlcolor=webbrown, linkcolor=pruss, citecolor=pruss, %pagecolor=RoyalBlue,%
    %urlcolor=Black, linkcolor=Black, citecolor=Black, %pagecolor=Black,%
    pdftitle={\Title},%
    pdfauthor={\textcopyright\ \Author, \University, \Department},%
    pdfsubject={},%
    pdfkeywords={},%
    pdfcreator={pdfLaTeX},%
    pdfproducer={LaTeX with hyperref and classicthesis}%
}

% ********************************************************************
% Setup autoreferences
% ********************************************************************
\makeatletter
\@ifpackageloaded{babel}%
    {%
       \addto\extrasamerican{%
			\renewcommand*{\figureautorefname}{Figure}%
			\renewcommand*{\tableautorefname}{Table}%
			\renewcommand*{\partautorefname}{Part}%
			\renewcommand*{\chapterautorefname}{Chapter}%
			\renewcommand*{\sectionautorefname}{Section}%
			\renewcommand*{\subsectionautorefname}{Section}%
			\renewcommand*{\subsubsectionautorefname}{Section}%
                }%
       \addto\extrasngerman{%
			\renewcommand*{\paragraphautorefname}{Absatz}%
			\renewcommand*{\subparagraphautorefname}{Unterabsatz}%
			\renewcommand*{\footnoteautorefname}{Fu\"snote}%
			\renewcommand*{\FancyVerbLineautorefname}{Zeile}%
			\renewcommand*{\theoremautorefname}{Theorem}%
			\renewcommand*{\appendixautorefname}{Anhang}%
			\renewcommand*{\equationautorefname}{Gleichung}%
			\renewcommand*{\itemautorefname}{Punkt}%
                }%
            % Fix to getting autorefs for subfigures right (thanks to Belinda Vogt for changing the definition)
            \providecommand{\subfigureautorefname}{\figureautorefname}%
    }{\relax}
\makeatother


% ********************************************************************
% Development Stuff
% ********************************************************************
\listfiles

% ********************************************************************
% Last, but not least...
% ********************************************************************
\usepackage{classicthesis}

\areaset[current]{312pt}{657pt}
\setlength{\marginparwidth}{9.5em}%
\setlength{\marginparsep}{2em}%

\usepackage{xargs}
\usepackage[colorinlistoftodos,prependcaption,textsize=tiny]{todonotes}
\newcommandx{\add}[2][1=]{\todo[inline, linecolor=red,backgroundcolor=red!25,bordercolor=red,#1]{#2}}
\newcommandx{\change}[2][1=]{\todo[inline,linecolor=blue,backgroundcolor=blue!25,bordercolor=blue,#1]{#2}}
\newcommandx{\info}[2][1=]{\todo[inline,linecolor=OliveGreen,backgroundcolor=OliveGreen!25,bordercolor=OliveGreen,#1]{#2}}
\newcommandx{\improvement}[2][1=]{\todo[inline,linecolor=Plum,backgroundcolor=Plum!25,bordercolor=Plum,#1]{#2}}
\newcommandx{\thiswillnotshow}[2][1=]{\todo[disable,#1]{#2}}

\makeatletter
\newcommand*{\textoverline}[1]{$\overline{\hbox{#1}}\m@th$}
\makeatother

\usepackage{tikz-picture}

% ****************************************************************************************************
% 8. Further adjustments (experimental)
% ****************************************************************************************************
% ********************************************************************
% Changing the text area
% ********************************************************************
\linespread{1.05} % a bit more for Palatino
\areaset[current]{312pt}{720pt} % 686 (factor 2.2) + 33 head + 42 head \the\footskip
\setlength{\marginparwidth}{9em}%
\setlength{\marginparsep}{2em}%

% ********************************************************************
% Using different fonts
% ********************************************************************
%\setmathfont{MnSymbol}
%\setmathfont[range=\mathup/{num,latin,Latin,greek,Greek}]{Minion Pro}
%\setmathfont[range=\mathbfup/{num,latin,Latin,greek,Greek}]{MinionPro-Bold}
%\setmathfont[range=\mathit/{num,latin,Latin,greek,Greek}]{MinionPro-It}
%\setmathfont[range=\mathbfit/{num,latin,Latin,greek,Greek}]{MinionPro-BoldIt}
%\setmathrm{Minion Pro}

%\usepackage{unicode-math}

%\setmainfont[Numbers={OldStyle,Proportional},SmallCapsFeatures={LetterSpace=6}]{Minion Pro}

%\usepackage[oldstylenums]{kpfonts} % oldstyle notextcomp
%\usepackage[osf]{libertine}
%\usepackage[light,condensed,math]{iwona}
%\renewcommand{\sfdefault}{iwona}
%\usepackage{lmodern} % <-- no osf support :-(
%\usepackage{cfr-lm} %
%\usepackage[urw-garamond]{mathdesign} %<-- no osf support :-(
%\usepackage[default,osfigures]{opensans} % scale=0.95
%\usepackage[sfdefault]{FiraSans}

\usepackage{pifont}

%\usepackage[math-style=upright]{unicode-math}
\usepackage{unicode-math}
%\usepackage[cmsy]{MnSymbol}
\setmainfont[Numbers={OldStyle,Proportional},SmallCapsFeatures={LetterSpace=6}]{Minion Pro}
\setmathfont[Scale=MatchLowercase]{STIX2Math.otf}
\setmonofont[Scale=MatchLowercase]{Inconsolata}

\usepackage{polyglossia}
\setmainlanguage{english}
\setotherlanguage{czech}

\usepackage{tocloft}
\renewcommand{\cftchapfont}{\normalfont\scshape}
\renewcommand{\cftpartfont}{\color{CTtitle}\normalfont\scshape}%
\setlength{\cftbeforechapskip}{7pt}
%\renewcommand\cftsecafterpnum{\vskip15pt}

% ****************************************************************************************************

\usepackage{sidenotes}

\newenvironment{summary}
{\bigskip\hrule\bigskip\itshape}
{\bigskip\hrule}
