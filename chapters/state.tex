\chapter{State of the art}
\label{ch:state}

% Současný stav řešené problematiky, přehled klasických i aktuálních výsledků a jejich porovnání, analýza problematiky vedoucí k vymezení oblasti zájmu budoucí disertační práce, 8 - 12 stran.


% Handbook of mode checking
%   - 3.2.1.4 Level of Abstraction
%   - 3.4.1.4 Data Type Abstraction:
% The choice of how to model the data type of the flit can have a big impact on the
% scalability of verification.
%
% Chapter 13:
% Abstraction tackles this challenge based on the assumption that a reduction of
% the information content results in a reduction of the size of the representation of
% a Kripke structure.
%
% Clanky:
% The topic of constructing abstractions is
% also one of the focuses of the theory of Abstract Interpretation [8, 37–40, 76], which
% is not treated in this chapter.

% TODO prerequisites

\section{Preliminaries}

This section introduces the notation used in the rest of the proposal.

\subsection{Programs, Control-Flow Graph, States}

We restrict to simplified version of \llvm (todo see).
% state space

% small step semantics

% llvm grammar

% abstract domain

% lattice ?

% galois connection

\section{Abstraction Framework}

%See handbook of model checkin chap. 13

%\subsection{Soundness}
% A principal requirement for any abstraction framework is that it is sound.


% see 13.4.1.1 Specific Abstractions: Examples
% 13.4.2 Additional Reading

\section{Abstraction base techniques}

\subsection{Abstract Interpretation}

\subsection{Abstraction-based model checking}

% reachability algorithm

\section{Abstract Domains}

% interval domain, sign domain, octagonal domain ...

\subsection{Unit domain}

% control flow of a program

\subsection{Term Domain}

\subsection{Predicate Abstraction}

\subsection{Refinement}

\section{Counterexample-guided abstraction refinement}

% algorithm

% picture

\section{Lazy abstraction}

\section{Eager Abstraction}

\section{Symbolic Abstraction}

\section{Interpolation}

% policy iteration
