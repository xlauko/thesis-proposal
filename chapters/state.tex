\chapter{State of the art}
\label{ch:state}

% Současný stav řešené problematiky, přehled klasických i aktuálních výsledků a jejich porovnání, analýza problematiky vedoucí k vymezení oblasti zájmu budoucí disertační práce, 8 - 12 stran.


% Handbook of mode checking
%   - 3.2.1.4 Level of Abstraction
%   - 3.4.1.4 Data Type Abstraction:
% The choice of how to model the data type of the flit can have a big impact on the
% scalability of verification.
%
% Chapter 13:
% Abstraction tackles this challenge based on the assumption that a reduction of
% the information content results in a reduction of the size of the representation of
% a Kripke structure.
%
% Clanky:
% The topic of constructing abstractions is
% also one of the focuses of the theory of Abstract Interpretation [8, 37–40, 76], which
% is not treated in this chapter.

% TODO prerequisites

\section{Preliminaries}
\label{sec:preliminaries}

This section introduces the notation used in the rest of the thesis proposal.
The presented notation is mainly based on the configurable software
verification technique \cite{Beyer2007, Beyer2018, Beyer2018b} and Cousot's
notation used to describe abstract interpretation \cite{Cousot2012}.

\subsection{Programs, Control-Flow Graph, States}

We restrict the presentation to a simplified version of \llvm programs
\cite{Lattner04}. An elementary unit of \llvm program is an instruction. We
distinguish binary instructions (arithmetic operations, relational operations),
memory access operations, data flow operations (branching instructions, calls,
return instructions). Because \llvmir is a typed language, it provides also
casting instructions. We admit scalar types (floats, integers), pointer types,
and aggregate types (arrays, structures).\sidenote{For detailed syntax and
semantics description see \autoref{ch:llvm}.}

The entire program in \llvm consists of global variables and functions.  A
function is a directed graph of basic blocks. A basic block is a sequence of
non-terminal instructions terminated by a terminal instruction.\sidenote{A
terminal instruction is either branching or return instruction.}

For a more accessible representation of an \llvm program, we will use a
\emph{control flow automaton} (CFA), which is a directed graph with \llvm
instructions attached to its edges (see \autoref{fig:cfa}).

\begin{definition}
    Given a set of instructions $\mathcal{I}$, a control flow automaton
    $\mathcal{A}$ is a tuple $(\mathcal{L}, l_{\textit{init}}, \mathcal{G})$,
    where $\mathcal{L}$ is a set of program locations, $l_{\textit{init}} \in
    \mathcal{L}$ is an initial location that represents a program entry point
    and $\mathcal{G} \subseteq (\mathcal{L} \times \mathcal{I} \times
    \mathcal{L})$ is set of edges between program locations, each labeled with
    an instruction that is executed when the control flows along the edge.
\end{definition}

The set of all variables that occur in the \llvm program is denoted by
$\mathcal{V}$.\sidenote{In addition to program variables, we will also consider
\llvm registers as variables. Even though \llvm registers are only assigned
ones in the \ssa.} A \emph{concrete state} $\sigma : \mathcal{V} \rightarrow
\mathsf{C}$ is a mapping from program variables to concrete values. We denote
the set of all concrete states as $\Sigma$. Thus, the concrete state can be
viewed as a state of program's memory that holds the value for each program
variable. A~set of admissible states at some location $l \in \mathcal{L}$ is
called a \emph{context} $\mathcal{C}_l \subseteq \mathcal{P}(\Sigma)$.

The semantics of instruction $i \in \mathcal{I}$ is defined by \emph{strongest
post condition} $SP_{i}(\cdot)$, i.e. it is executed on a particular program
state a accordingly updates values of program variables -- for further details
see \autoref{ch:llvm}.

A set of all edges $\mathcal{G}$ induces a transition relation $\rightarrow$ on
the set of states $\Sigma$, such that for each $g \in \mathcal{G}$ there is
$\xrightarrow{g} \: \subseteq \Sigma \times \{g\} \times \Sigma$. The
transition relation is a union over all edges $\rightarrow \bigcup_{g
\in\mathcal{G}} \xrightarrow{g}$.

A \emph{program path} $\pi$ is a sequence of consecutive edges $\langle
\rightarrow_1, \rightarrow_2, \dots, \rightarrow_n \rangle$ such that all edges
form in continuous path in transition system induced by $\mathcal{G}$.  A path
is called a \emph{program path} if it starts from initial program location
$\l_{\textit{init}}$. The semantics of path $\pi$ is defined by iterative
application of $SP_{i}( \cdot )$ for each instruction in path $\pi$. Given
initial constraints $\phi$ on variables, the strongest postcondition of path
$\pi$ is $\textsf{SP}_{\pi} = \textsf{SP}_{i_n}(\dots,SP_{i_1}(\phi),\dots)$. A
path $\pi$ is called \emph{feasible} if $SP_{\pi}(\textit{true})$ is
satisfiable and \emph{infeasible} otherwise \cite{Beyer2018b}.

A \emph{verification task} for  CFA $\mathcal{A} = (\mathcal{L},
l_{\textit{init}}, \mathcal{G})$ is to show that error location
$l_{\textit{err}} \in \mathcal{L}$ is unreachable in $\mathcal{A}$, or to find
feasible error path.

\begin{figure}

\begin{minipage}[t]{0.5\textwidth}
\begin{minted}[linenos]{llvm}
define i32 @main() {
entry:
  %x = alloca i32
  store i32 0 to i32* %x
  br label %loop
loop:
  %v = load i32* %x
  %a = add i32 %v, 1
  store i32 %a to i32* %x
  %b = icmp ult i32 %a, 5
  br %b, label %loop, label %end
end:
  ret i32 0
}
\end{minted}
\end{minipage}
\hfill
\begin{minipage}[t]{0.48\textwidth}
\strut\vspace*{-\baselineskip}\newline\centering
\begin{tikzpicture}[node distance=1.1em]
    \node [loc] (l1) {$l_1$};
    \node [lab, left = 0cm of l1] (entry) {\llvmint{entry:}};

    \node [loc, below = of l1] (l2) {$l_2$};
    \node [loc, below = of l2] (l3) {$l_3$};
    \node [lab, left = 0cm of l3] (loop) {\llvmint{loop:}};

    %\node[fit=(l1) (l2), draw, dashed] (ebb) {};

    \node [loc, below = of l3] (l4) {$l_4$};
    \node [loc, below = of l4] (l5) {$l_5$};
    \node [loc, below = of l5] (l6) {$l_6$};
    \node [loc, below = of l6] (l7) {$l_7$};

    %\node[fit=(l3) (l4) (l5) (l6) (l7), draw, dashed] (lbb) {};

    \node [loc, below = of l7] (l8) {$l_8$};
    \node [lab, left = 0cm of l8] (loop) {\llvmint{end:}};

    \node [loc, below = of l8] (l9) {$l_9$};

    %\node[fit=(l8) (l9), draw, dashed] (lbb) {};

    \draw [->, >=stealth] (l1) -- node[midway, right] {\llvmint{|\var{\%x}| = alloca i32}} (l2);
    \draw [->, >=stealth] (l2) -- node[midway, right] {\llvmint{store 0 to |\var{\%x}|}} (l3);
    \draw [->, >=stealth] (l3) -- node[midway, right] {\llvmint{|\var{\%v}| = load |\var{\%x}|}} (l4);
    \draw [->, >=stealth] (l4) -- node[midway, right] {\llvmint{|\var{\%a}| = add |\var{\%v}| 1}} (l5);
    \draw [->, >=stealth] (l5) -- node[midway, right] {\llvmint{store |\var{\%a}| to |\var{\%x}|}} (l6);
    \draw [->, >=stealth] (l6) -- node[midway, right] {\llvmint{|\var{\%b}| = ult |\var{\%a}| 5}} (l7);
    \draw [->, >=stealth] (l7) -- node[midway, right] {\llvmint{[|\var{\%b}| = false]}} (l8);
    \draw [->, >=stealth] (l7) to [bend left=40] node[above, rotate = 90] {\llvmint{[|\var{\%b}|= true]}} (l3);
    \draw [->, >=stealth] (l8) -- node[midway, right] {\llvmint{ret 0}} (l9);

\end{tikzpicture}
\end{minipage}

\caption{An example of \llvm program (left) and its CFA (right).}
\label{fig:cfa}
\end{figure}

\section{Abstract Interpretation}
\label{sec:abstraction}

The original idea of abstract interpretation dates back to the late 70s, first
summarized by Patrick and Radhia Cousot \cite{Cousot1977}.
They describe \emph{abstract interpretation} as a theory of abstraction and
constructive approximation of the mathematical structures used in the formal
description of programming languages and the inference or verification of
undecidable program properties~\cite{Cousot2012}.

For our course, the goal of abstract interpretation is to assign each program
location a context in a given domain, i.e., given a set of program states, the
abstraction represents properties of these states by abstract states. Program
contexts are computed by solving a system of fixpoint equations generated from
a transition function \cite{Cousot1977}. Such computation simulates the
execution of a program in an abstract domain. As a result, each abstract context represent possible properties (invariants about states) of variables at a location after the execution of the program
with arbitrary input. The states\sidenote{State of computation comprise of all
intermediate contexts assigned to program locations.} of the computation forms
a complete lattice: $(S, \sqcap, \sqcup, \sqsubseteq, \top, \bot)$, where $S$
is a set of states, $\sqcup$, $\sqcap$ are meet and join operators,
$\sqsubseteq$ ordering of a lattice, $\top$ is the greatest element and $\bot$
it the least element of the lattice. Throughout the rest of the thesis proposal
we will assume familiarity with the basic notions of lattice theory
\cite{Birkhoff1940}.

Given the concrete contexts from previous definitions, the \emph{concrete
semantic domain} is a complete lattice $(\mathcal{P}(\Sigma), \cap, \cup,
\subseteq, \Sigma, \emptyset)$. Even though this technique gives us a tool to
compute all possible states that a program might reach, it is, unfortunately,
uncomputable (it can be easily shown by reduction to halting problem). To
mitigate the problem of computability, we employ an abstract domain instead of
a concrete one. In the following sections, we will investigate various types of
abstract domains with their advantages and disadvantages.

\section{Abstract Domains}
\label{sec:domains}

To obtain a computable model, we need to drop some information about program
variables. For instance, to drop all the information about each program
variable except whether it is positive, negative or zero. This can be achieved
by replacement of values domain we are computing on $\mathsf{C}$ by an abstract
domain $\domainm{sign} = \{ \bot, -, 0, +, \top)$ =, where $\bot$ represents an
undefined value (variable without assigned value), $+$,$-$ and $0$ denote a
\begin{marginfigure}%
    \centering
    \begin{tikzpicture}[node distance=1em]
    \node [] (t) {$\top$};
    \node [below = of t] (0) {$0$};
    \node [left = of 0] (m) {$-$};
    \node [right = of 0] (p) {$+$};
    \node [below = of 0] (b) {$\bot$};
    \draw [thin] (t) -- (m) -- (b);
    \draw [thin] (t) -- (0) -- (b);
    \draw [thin] (t) -- (p) -- (b);
    \end{tikzpicture}
    \caption{$\mathsf{A}_{\textit{sign}}$ domain lattice.}
    \label{fig:signd}%
\end{marginfigure}%
sign of variable and $\top$ is for an arbitrary value (i.e., variable can be
either negative, positive or zero).

\begin{figure}%
\begin{minipage}[t]{0.3\textwidth}
\strut\vspace*{-\baselineskip}\newline\centering
\resizebox{\textwidth}{!}{
\begin{tikzpicture}[node distance=1.1em]
    \node [loc] (l1) {$l_1$};
    \node [lab, left = 0cm of l1] (entry) {\llvmint{entry:}};
    \node [lab, right = 0cm of l1] (e1) {$(\bot, \bot, \bot, \bot)$};

    \node [loc, below = of l1] (l2) {$l_2$};
    \node [lab, right = 0cm of l2] (e2) {$(\bot, \bot, \bot, \bot)$};
    \node [loc, below = of l2] (l3) {$l_3$};
    \node [lab, left = 0cm of l3] (loop) {\llvmint{loop:}};
    \node [lab, right = 0cm of l3] (e3) {$(0, \bot, \bot, \bot)$};

    %\node[fit=(l1) (l2), draw, dashed] (ebb) {};

    \node [loc, below = of l3] (l4) {$l_4$};
    \node [lab, right = 0cm of l4] (e4) {$(\bot, \bot, \bot, \bot)$};
    \node [loc, below = of l4] (l5) {$l_5$};
    \node [lab, right = 0cm of l5] (e5) {$(\bot, \bot, \bot, \bot)$};
    \node [loc, below = of l5] (l6) {$l_6$};
    \node [lab, right = 0cm of l6] (e6) {$(\bot, \bot, \bot, \bot)$};
    \node [loc, below = of l6] (l7) {$l_7$};
    \node [lab, right = 0cm of l7] (e7) {$(\bot, \bot, \bot, \bot)$};

    %\node[fit=(l3) (l4) (l5) (l6) (l7), draw, dashed] (lbb) {};

    \node [loc, below = of l7] (l8) {$l_8$};
    \node [lab, left = 0cm of l8] (loop) {\llvmint{end:}};
    \node [lab, right = 0cm of l8] (e8) {$(\bot, \bot, \bot, \bot)$};

    \node [loc, below = of l8] (l9) {$l_9$};
    \node [lab, right = 0cm of l9] (e9) {$(\bot, \bot, \bot, \bot)$};

    %\node[fit=(l8) (l9), draw, dashed] (lbb) {};

    \draw [->, >=stealth] (l1) -- node[midway, right] {\llvmint{|\var{\%x}| = alloca i32}} (l2);
    \draw [->, >=stealth] (l2) -- node[midway, right] {\llvmint{store 0 to |\var{\%x}|}} (l3);
    \draw [->, >=stealth] (l3) -- node[midway, right] {\llvmint{|\var{\%v}| = load |\var{\%x}|}} (l4);
    \draw [->, >=stealth] (l4) -- node[midway, right] {\llvmint{|\var{\%a}| = add |\var{\%v}| 1}} (l5);
    \draw [->, >=stealth] (l5) -- node[midway, right] {\llvmint{store |\var{\%a}| to |\var{\%x}|}} (l6);
    \draw [->, >=stealth] (l6) -- node[midway, right] {\llvmint{|\var{\%b}| = ult |\var{\%a}| 5}} (l7);
    \draw [->, >=stealth] (l7) -- node[midway, right] {\llvmint{[|\var{\%b}| = false]}} (l8);
    \draw [->, >=stealth] (l7) to [bend left=40] node[above, rotate = 90] {\llvmint{[|\var{\%b}|= true]}} (l3);
    \draw [->, >=stealth] (l8) -- node[midway, right] {\llvmint{ret 0}} (l9);
\end{tikzpicture}
}
\end{minipage}
\hfill
\begin{minipage}[t]{0.3\textwidth}
\strut\vspace*{-\baselineskip}\newline\centering
\resizebox{\textwidth}{!}{
\begin{tikzpicture}[node distance=1.1em]
    \node [loc] (l1) {$l_1$};
    \node [lab, left = 0cm of l1] (entry) {\llvmint{entry:}};
    \node [lab, right = 0cm of l1] (e1) {$(\bot, \bot, \bot, \bot)$};

    \node [loc, below = of l1] (l2) {$l_2$};
    \node [lab, right = 0cm of l2] (e2) {$(\bot, \bot, \bot, \bot)$};
    \node [loc, below = of l2] (l3) {$l_3$};
    \node [lab, left = 0cm of l3] (loop) {\llvmint{loop:}};
    \node [lab, right = 0cm of l3] (e3) {$(0, \bot, \bot, \bot)$};

    %\node[fit=(l1) (l2), draw, dashed] (ebb) {};

    \node [loc, below = of l3] (l4) {$l_4$};
    \node [lab, right = 0cm of l4] (e4) {$(0, 0, \bot, \bot)$};
    \node [loc, below = of l4] (l5) {$l_5$};
    \node [lab, right = 0cm of l5] (e5) {$(0, 0, +, \bot)$};
    \node [loc, below = of l5] (l6) {$l_6$};
    \node [lab, right = 0cm of l6] (e6) {$(+, 0, +, \bot)$};
    \node [loc, below = of l6] (l7) {$l_7$};
    \node [lab, right = 0cm of l7] (e7) {$(+, 0, +, \top)$};

    %\node[fit=(l3) (l4) (l5) (l6) (l7), draw, dashed] (lbb) {};

    \node [loc, below = of l7] (l8) {$l_8$};
    \node [lab, left = 0cm of l8] (loop) {\llvmint{end:}};
    \node [lab, right = 0cm of l8] (e8) {$(\bot, \bot, \bot, \bot)$};

    \node [loc, below = of l8] (l9) {$l_9$};
    \node [lab, right = 0cm of l9] (e9) {$(\bot, \bot, \bot, \bot)$};

    %\node[fit=(l8) (l9), draw, dashed] (lbb) {};

    \draw [->, >=stealth] (l1) -- node[midway, right] {\llvmint{|\var{\%x}| = alloca i32}} (l2);
    \draw [->, >=stealth] (l2) -- node[midway, right] {\llvmint{store 0 to |\var{\%x}|}} (l3);
    \draw [->, >=stealth] (l3) -- node[midway, right] {\llvmint{|\var{\%v}| = load |\var{\%x}|}} (l4);
    \draw [->, >=stealth] (l4) -- node[midway, right] {\llvmint{|\var{\%a}| = add |\var{\%v}| 1}} (l5);
    \draw [->, >=stealth] (l5) -- node[midway, right] {\llvmint{store |\var{\%a}| to |\var{\%x}|}} (l6);
    \draw [->, >=stealth] (l6) -- node[midway, right] {\llvmint{|\var{\%b}| = ult |\var{\%a}| 5}} (l7);
    \draw [->, >=stealth] (l7) -- node[midway, right] {\llvmint{[|\var{\%b}| = false]}} (l8);
    \draw [->, >=stealth] (l7) to [bend left=40] node[above, rotate = 90] {\llvmint{[|\var{\%b}|= true]}} (l3);
    \draw [->, >=stealth] (l8) -- node[midway, right] {\llvmint{ret 0}} (l9);
\end{tikzpicture}
}
\end{minipage}
\hfill
\begin{minipage}[t]{0.3\textwidth}
\strut\vspace*{-\baselineskip}\newline\centering
\resizebox{\textwidth}{!}{
\begin{tikzpicture}[node distance=1.1em]
    \node [loc] (l1) {$l_1$};
    \node [lab, left = 0cm of l1] (entry) {\llvmint{entry:}};
    \node [lab, right = 0cm of l1] (e1) {$(\bot, \bot, \bot, \bot)$};

    \node [loc, below = of l1] (l2) {$l_2$};
    \node [lab, right = 0cm of l2] (e2) {$(\bot, \bot, \bot, \bot)$};
    \node [loc, below = of l2] (l3) {$l_3$};
    \node [lab, left = 0cm of l3] (loop) {\llvmint{loop:}};
    \node [lab, right = 0cm of l3] (e3) {$(\top, \top, \top, \top)$};

    %\node[fit=(l1) (l2), draw, dashed] (ebb) {};

    \node [loc, below = of l3] (l4) {$l_4$};
    \node [lab, right = 0cm of l4] (e4) {$(\top, \top, \top, \top)$};
    \node [loc, below = of l4] (l5) {$l_5$};
    \node [lab, right = 0cm of l5] (e5) {$(\top, \top, \top, \top)$};
    \node [loc, below = of l5] (l6) {$l_6$};
    \node [lab, right = 0cm of l6] (e6) {$(\top, \top, \top, \top)$};
    \node [loc, below = of l6] (l7) {$l_7$};
    \node [lab, right = 0cm of l7] (e7) {$(\top, \top, \top, \top)$};

    %\node[fit=(l3) (l4) (l5) (l6) (l7), draw, dashed] (lbb) {};

    \node [loc, below = of l7] (l8) {$l_8$};
    \node [lab, left = 0cm of l8] (loop) {\llvmint{end:}};
    \node [lab, right = 0cm of l8] (e8) {$(\top, \top, \top, \top)$};

    \node [loc, below = of l8] (l9) {$l_9$};
    \node [lab, right = 0cm of l9] (e9) {$(\top, \top, \top, \top)$};

    %\node[fit=(l8) (l9), draw, dashed] (lbb) {};

    \draw [->, >=stealth] (l1) -- node[midway, right] {\llvmint{|\var{\%x}| = alloca i32}} (l2);
    \draw [->, >=stealth] (l2) -- node[midway, right] {\llvmint{store 0 to |\var{\%x}|}} (l3);
    \draw [->, >=stealth] (l3) -- node[midway, right] {\llvmint{|\var{\%v}| = load |\var{\%x}|}} (l4);
    \draw [->, >=stealth] (l4) -- node[midway, right] {\llvmint{|\var{\%a}| = add |\var{\%v}| 1}} (l5);
    \draw [->, >=stealth] (l5) -- node[midway, right] {\llvmint{store |\var{\%a}| to |\var{\%x}|}} (l6);
    \draw [->, >=stealth] (l6) -- node[midway, right] {\llvmint{|\var{\%b}| = ult |\var{\%a}| 5}} (l7);
    \draw [->, >=stealth] (l7) -- node[midway, right] {\llvmint{[|\var{\%b}| = false]}} (l8);
    \draw [->, >=stealth] (l7) to [bend left=40] node[above, rotate = 90] {\llvmint{[|\var{\%b}|= true]}} (l3);
    \draw [->, >=stealth] (l8) -- node[midway, right] {\llvmint{ret 0}} (l9);
\end{tikzpicture}
}
\end{minipage}
    \caption{\domain{sign} fixpoint iteration for variables
    (\llvmint{|\var{\%x}, \var{\%v}, \var{\%a}, \var{\%b}|}). A~tuple at each
    location represents an abstract evaluation of the variables in this order.
    The presented states of iteration are after first iteration
    (\textbf{left}), after 5 iterations (\textbf{middle}), reached fixpoint
    (\textbf{right}).}
    \label{fig:signcomp}%
    \add{todo page width}
\end{figure}%

We say that $\mathsf{A}$ is an abstraction of $\mathsf{C}$ because it carries
less information about program variables. One requirement on $\mathsf{A}$ is
that its values form a complete lattice (see \autoref{fig:signd}). Moreover, a
well-defined domain is required to have an image for each concrete value and
that $\mathcal{A}$ is over approximating abstraction. Such properties of
domains are described by \textbf{\emph{Galois connection}}. For concrete
lattice $(C, \subseteq)$ and abstract lattice $(A, \sqsubseteq)$ a galois
connection is $(\alpha, \gamma)$, where $\alpha : C \rightarrow A$, $\gamma : A
\rightarrow C$ are monotonic functions such that:

\bigskip
\begin{center}
\begin{minipage}{0.45\textwidth}
\centering
    \emph{Abstraction \\ over-approximates}
\begin{tikzpicture}
    \node [domainbox, thick, color = apple, label=\color{apple}$C$] (c) {};
    \node [context, below = 1em of c.north] (cs) {};
    \node [below = 0.5em of cs.north] (co) {\textbullet$c$};
    \node [domainbox, thick, color = vivid, right = of c, label=\color{vivid}$A$] (a) {};
    \node [context, below = 1em of a.north] (as) {};

    \draw [->, >=stealth] (as) to [bend right=40] node[above] {$\gamma$} (cs);
    \draw [->, >=stealth] (co) to [bend right=40] node[below] {$\alpha$} (as);
\end{tikzpicture}
    \[
        \forall c \in C.\: c \subseteq \gamma( \alpha( c ) )
    \]
\end{minipage}
\hfill
\begin{minipage}{0.45\textwidth}
\centering
    \emph{Abstraction after concretisation yields no imprecision}
\begin{tikzpicture}
    \node [domainbox, thick, color = apple, label=\color{apple}$C$] (c) {};
    \node [context, below = 1em of c.north] (cs) {};
    \node [domainbox, thick, color = vivid, right = of c, label=\color{vivid}$A$] (a) {};
    \node [below = 0.5em of as.north] (ao) {$a$};
    \node [context, below = 1em of a.north] (as) {};

    \draw [->, >=stealth] (as) to [bend right=40] node[above] {$\gamma$} (cs);
    \draw [->, >=stealth] (cs) to [bend right=40] node[below] {$\alpha$} (ao);
\end{tikzpicture}
    \[
        \forall a \in A.\: \alpha( \gamma( a ) ) \sqsubseteq a
    \]
\end{minipage}
\end{center}

\noindent
In abstract interpretation $C$ and $A$ are called \emph{concrete} respectivelly
\emph{abstract} domain and they are assumed to be complete lattices. We call
$\alpha$ the \emph{abstraction} map and $\gamma$ the \emph{concretization} map.

Given an abstract domain and abstract operations\sidenote{In our case a
definition how \llvm instructions transform abstract values.} we can perform a
fixpoint iteration\sidenote{For detailed algorithm see \cite{Cousot1977}} a
by Knaster-Tarski theorem \cite{Tarski1955} we are guaranteed to obtain a
solution, see \autoref{fig:signcomp}.

\begin{marginfigure}%
    \centering
\resizebox{\textwidth}{!}{
    \begin{tikzpicture}[node distance=1em]
    \node [] (t) {$\top$};
    \node [below = 3em of t] (0) {$0$};
    \node [left = of 0] (m1) {$-1$};
    \node [left = of m1] (m2) {$-2$};
    \node [left = of m2] (m) {$\dots$};
    \node [right = of 0] (p1) {$1$};
    \node [right = of p1] (p2) {$2$};
    \node [right = of p2] (p) {$\dots$};
    \node [below = 3em of 0] (b) {$\bot$};
    \draw [thin] (t) -- (m) -- (b);
    \draw [thin] (t) -- (m2) -- (b);
    \draw [thin] (t) -- (m1) -- (b);
    \draw [thin] (t) -- (0) -- (b);
    \draw [thin] (t) -- (p1) -- (b);
    \draw [thin] (t) -- (p2) -- (b);
    \draw [thin] (t) -- (p) -- (b);
    \end{tikzpicture}
}
    \caption{\domain{cp} domain lattice.}
    \label{fig:cp}%
\end{marginfigure}%

Even though our example program in \autoref{fig:signcomp} is quite trivial, the
sign domain \domain{sign} was not able to preserve any interesting properties,
and all resulting contexts are either $\top$ or $\bot$.  Sign domain belongs to
the class of simple domains that reason only about single property of variables --
also known as \emph{property domains}. Another famous example of a property
domain is a \textbf{\emph{parity domain}}, which captures whether the
variable is odd or even. The parity domain is just a special case of a
\textbf{\emph{congruence domain}} \cite{Granger1989, Granger1991}.

The basic building block of these domains is \emph{basis} that expresses how to
abstract one variable and has abstract counterparts for arithmetic operators
instead of abstract counterparts for complex transfer functions. Thanks to
galois connection the rest of interpretation can be derived automatically
\cite{Mine2004Thesis}.

A~\textbf{\emph{constant propagation domain}} \domain{cp} (see
\autoref{fig:cp}) is another example of property domain (i.e., variable can
gain only single value) used in optimization techniques \cite{Kildall1973}.  It
is possible to combine multiple domains to reason about multiple properties
either by tracking each property separately or by reduced product of domains
which is more precise and more efficient than multiple separate interpretations
\cite{Cousot2011b} (see \autoref{fig:cpsign}). For further information see
\autoref{sec:domainrefinement}.

\begin{marginfigure}%
    \centering
\resizebox{\textwidth}{!}{
    \begin{tikzpicture}[node distance=1em]
    \node [] (t) {$\top$};
    \node [below = 6em of t] (0) {$0$};
    \node [left = of 0] (m1) {$-1$};
    \node [left = of m1] (m2) {$-2$};
    \node [above = 3em of m2] (mm) {$-$};
    \node [left = of m2] (m) {$\dots$};
    \node [right = of 0] (p1) {$1$};
    \node [right = of p1] (p2) {$2$};
    \node [right = of p2] (p) {$\dots$};
    \node [above = 3em of p2] (pp) {$+$};
    \node [below = 3em of 0] (b) {$\bot$};
        \draw [thin] (t) -- (mm) -- (m)  -- (b);
        \draw [thin] (t) -- (mm) -- (m2) -- (b);
        \draw [thin] (t) -- (mm) -- (m1) -- (b);
    \draw [thin] (t) -- (0) -- (b);
        \draw [thin] (t) -- (pp)-- (p1) -- (b);
        \draw [thin] (t) -- (pp)-- (p2) -- (b);
        \draw [thin] (t) -- (pp) -- (p) -- (b);
    \end{tikzpicture}
}
    \caption{\domain{cs} is a joint sign (\domain{sign}) and constant propagation (\domain{cp}) domain.}
    \label{fig:cpsign}%
\end{marginfigure}%

Probably the simplest of domains is a definedness domain \domain{D}, which only
tracks a definedness of values (see \autoref{fig:unitd}). Despite its
simplicity, it can be useful to detect reachability of program locations. Even
simpler domain is a single value domain \domain{U}. The single value domain is
usefull if we want to omit a variable from the model of the program.

\begin{marginfigure}%
    \centering
    \begin{tikzpicture}[node distance=1em]
    \node [] (t) {$\top$};
    \node [below = of t] (b) {$\bot$};
    \draw [thin] (t) -- (b);
    \end{tikzpicture}
    \caption{\domain{U} tracks only definedness of variables.}
    \label{fig:unitd}%
\end{marginfigure}%

Since the precision of an abstract domain highly depends on the program
structure, many abstract domains were designed to track various program
properties. Let us take it chronologically. The first domain presented in
Cousot's paper \cite{Cousot1977} to achieve more precise results was an
\textbf{\emph{interval domain}} \domain{I}. In the interval abstraction a set of values
is represented by its minimal and maximal value. Since in some cases, it is not
possible to determine the boundary values, the interval domain utilizes
infinities to denote arbitrary bound -- we represent arbitrary value
$\top_{\textit{I}}$ by $[-\infty, \infty]$ and undetermined value $\bot_{\textit{I}}$
by empty interval $\emptyset$.

\begin{marginfigure}
\strut\vspace*{-\baselineskip}\newline\centering
\resizebox{\textwidth}{!}{
\begin{tikzpicture}[node distance=1.1em]
    \node [loc] (l1) {$l_1$};
    \node [lab, left = 0cm of l1] (entry) {\llvmint{entry:}};
    \node [lab, right = 0cm of l1] (e1) {$(\bot, \bot, \bot, \bot)$};

    \node [loc, below = of l1] (l2) {$l_2$};
    \node [lab, right = 0cm of l2] (e2) {$(\bot, \bot, \bot, \bot)$};
    \node [loc, below = of l2] (l3) {$l_3$};
    \node [lab, left = 0cm of l3] (loop) {\llvmint{loop:}};
    \node [lab, right = 0cm of l3] (e3) {$([0,0], [0,0], [0,0], [0,0])$};

    %\node[fit=(l1) (l2), draw, dashed] (ebb) {};

    \node [loc, below = of l3] (l4) {$l_4$};
    \node [lab, right = 0cm of l4] (e4) {$([0,0], [0,0], [1,1], [0,1])$};
    \node [loc, below = of l4] (l5) {$l_5$};
    \node [lab, right = 0cm of l5] (e5) {$([0,0], [0,0], [1,1], [0,1])$};
    \node [loc, below = of l5] (l6) {$l_6$};
    \node [lab, right = 0cm of l6] (e6) {$([1,1], [0,0], [1,1], [0,1])$};
    \node [loc, below = of l6] (l7) {$l_7$};
    \node [lab, right = 0cm of l7] (e7) {$([1,1], [0,0], [1,1], [0,1])$};

    %\node[fit=(l3) (l4) (l5) (l6) (l7), draw, dashed] (lbb) {};

    \node [loc, below = of l7] (l8) {$l_8$};
    \node [lab, left = 0cm of l8] (loop) {\llvmint{end:}};
    \node [lab, right = 0cm of l8] (e8) {$([0,0], [0,0], [0,0], [0,0])$};

    \node [loc, below = of l8] (l9) {$l_9$};
    \node [lab, right = 0cm of l9] (e9) {$([0,0], [0,0], [0,0], [0,0])$};

    %\node[fit=(l8) (l9), draw, dashed] (lbb) {};

    \draw [->, >=stealth] (l1) -- node[midway, right] {\llvmint{|\var{\%x}| = alloca i32}} (l2);
    \draw [->, >=stealth] (l2) -- node[midway, right] {\llvmint{store 0 to |\var{\%x}|}} (l3);
    \draw [->, >=stealth] (l3) -- node[midway, right] {\llvmint{|\var{\%v}| = load |\var{\%x}|}} (l4);
    \draw [->, >=stealth] (l4) -- node[midway, right] {\llvmint{|\var{\%a}| = add |\var{\%v}| 1}} (l5);
    \draw [->, >=stealth] (l5) -- node[midway, right] {\llvmint{store |\var{\%a}| to |\var{\%x}|}} (l6);
    \draw [->, >=stealth] (l6) -- node[midway, right] {\llvmint{|\var{\%b}| = ult |\var{\%a}| 5}} (l7);
    \draw [->, >=stealth] (l7) -- node[midway, right] {\llvmint{[|\var{\%b}| = false]}} (l8);
    \draw [->, >=stealth] (l7) to [bend left=40] node[above, rotate = 90] {\llvmint{[|\var{\%b}|= true]}} (l3);
    \draw [->, >=stealth] (l8) -- node[midway, right] {\llvmint{ret 0}} (l9);
\end{tikzpicture}
}
    \caption{TODO correct intervals: Interval abstraction example after fixpoint iteration.}
\end{marginfigure}
\add{ example of interval abstraction on example program }

Unfortunately, with the infinite lattice of the interval
domain,\sidenote{The interval domain does not meet ascending chain condition.} the
abstract interpretation obtained a possible infinite or impractically long
executions (e.g., when the interpretation can not determine the bound of a loop
it can increment the boundary of an interval indefinitely never reaching the
fixpoint). For this reason, abstract interpreters employ widening and
narrowing techniques to accelerate convergence \cite{Cousot1992a, Cortesi2011}.
The idea of widening is to overshoot the least fixpoint after few
unsuccessful iterations of the interpretation and subsequently by narrowing to
refine the over-approximated solution.
\begin{definition}
    \textbf{Widening operator} $\nabla$ for domain $(\mathcal{A},
    \sqsubseteq)$ fulfills:
    \begin{enumerate}
        \item $\forall v_1, v_2 \in \mathcal{A} : v_1 \sqcup v_2 \sqsubseteq v_1 \nabla v_2$,
        \item for each sequence $(v_k)_{k \in \mathbb{N}}$, the sequence $(v^{\nabla}_k)_{k \in \mathbb{N}}$ defined as $v^{\nabla}_0 = v_0$ and $v^{\nabla}_k = v^{\nabla}_{k-1} \nabla v_k$ reaches a fixpoint after finitely many steps.
    \end{enumerate}
\end{definition}
\begin{definition}
    \textbf{Narrowing operator} $\Delta$ for domain $(\mathcal{A},
    \sqsubseteq)$ fulfills:
    \begin{enumerate}
        \item $\forall v_1, v_2 \in \mathcal{A} : v_1 \sqsupseteq v_2 \implies v_1 \Delta v_2 \sqsubseteq v_2$,
        \item for each sequence $(v_k)_{k \in \mathbb{N}}$, the sequence $(v^{\Delta}_k)_{k \in \mathbb{N}}$ defined as $v^{\Delta}_0 = v_0$ and $v^{\Delta}_k = v^{\Delta}_{k-1} \Delta v_k$ reaches a fixpoint after finitely many steps.
    \end{enumerate}
\end{definition}

By different implementation of widening and narrowing, abstract interpreters
employ a different strategies to achieve convergence -- for example widening
with thresholds \cite{Blanchet2003, Lakhdar2011}, delayed widening, parma widening
\cite{Bagnara2008} or abstract acceleration \cite{Gonnord2006, Feautrier2010}.

An alternative approach to tackle infinite interpretation that does not require
any extrapolation operator is a \textbf{\emph{policy iteration}}
\cite{Costan2005, Gaubert2007, Gawlitza2007, Gawlitza2007b, Gawlitza2011}. The
idea of policies is to compute fixpoint solution of a sequence of simpler
semantic equations, such that the least fixpoint is reached after a finite
number of iterations.  The sequence of policies defines a strategy to approach
the fixpoint either from above or below. Policies are formed from a
decomposition of original abstraction.

\begin{marginfigure}
\begin{minted}[linenos]{c}
int i = 10;
int v = 0;
while (i >= 0) {
    i = i - 1;
    if (random())
        v = v + 1;
}
\end{minted}
    \caption{Program that requires a relational invariant.}
    \label{fig:relationalc}
\end{marginfigure}

The drawback of until now presented domains is that they are not able to track
relational properties between variables -- hence we address them as
\emph{non-relational} abstract domains. Given \autoref{fig:relationalc}, to
show that $v \leq 11$ at line 7, we need to prove a relational loop invariant
$v + i \leq 10$.

%\add{other applications of relational domains: analysis of programs with
%symbolic parameters, modular analysis of procedures, inference of non-uniform
%non-numerical invariants (pointer analysis)}

\subsection{Relational abstract domains}

The simplest of the relational abstract domains is a \emph{linear equality
domain} that captures information about affine relationships among program
variables \cite{Karr1976}. In this domain, an abstract variable is represented
by the affine subspace of program state space. The problem with the affine
domain is in its nonunique minimal form; this problem is solved by combination
with the congruence domain \cite{Granger1991}.

Another relational domain is a \emph{polyhedron abstract domain} which is
inspired by the theory of linear programming and optimization \cite{Cousot1978}
and provides a unique representation. Though, the manipulation with polyhedron
abstract values involves costly computation of a simplex algorithm
\cite{Schrijver1986}. Nowadays a polyhedron abstract domain is still
extensively studied -- an abstraction of polyhedron by bounding boxes to
simplify abstract operations \cite{Singh2017}, or in combination with interval
domain to reason about interval linear properties \cite{Chen2009}.

\subsection{Weakly relational abstract domains}

In the middle between relational and non-relational abstract domains stand
\textbf{\emph{weakly relational abstract domains}}. These domains aim to
increase the precision of non-relational domains (interval domain) and reduce
the cost of abstraction of relational domains (polyhedron domain) by tracking
only a reduced amount of relations.

The notable representant of weakly-relational domains is an \emph{octagon
abstract domain} \cite{Mine2006}. This domain efficiently enables to represent
invariants of form $\pm x \pm y \leq c$, where $x$ and $y$ are program
variables and $c$ is a constant.  Moreover, the advantage of the octagon domain
is in its representation by potential graphs and difference-bound matrices
\cite{Larsen1997}, which allows a unique minimal representation of abstract
values. The octagon domain was further optimized by bucketing of octagons
\cite{Blanchet2003, Venet2004} and modified to extend captured properties
\cite{Claris2004, Mine2004}. Further restriction provides a zone domain
\cite{Mine2001} which reasons only about invariants of form $x - y \leq c$ and
$\pm x \leq c$.

A less precise and lightweight relational domain of pentagons was introduced in
\cite{Logozzo2010} with applicability on validation of array manipulating
byte-code. The pentagon domain captures properties of the form of $x \in [a,b]
\wedge x < y$.

A recent abstract domain utilizes parallelotopes to encode any linear
constraint as the polyhedra, while it preserves the efficiency of weakly
relational abstract domains \cite{Amato2017}.

A disadvantageous aspect of relational abstract domains is their complicated
representation and therefore computationally less efficient abstract
operations. In addition to maintaining more properties about variables, the
representation also needs to assure uniqueness of the representation. Hence the
operations need to incorporate some sort of normalization (e.g., difference
bound matrices in the octagon domain \cite{Mine2006}, or generators of affine
space in the linear equality domain \cite{Karr1976}).

Except picking of a suitable abstraction, a precision of abstract methods can
be improved by symbolic methods. Miné presents in his work \cite{Mine2006b} a
set of lightweight simplifications (linearization and symbolic constant
propagation) which are used to transform numerical expressions in order to
increase the precision of the analysis.

\subsubsection{Predicate Abstraction}

\subsubsection{Term Domain}

\prule
\bigskip

So far, we have presented only domains, which abstract only scalar values -- we call
them numerical abstract domains.  Overall a \emph{numerical abstract domain} is
determined by three ingredients:
\begin{enumerate}
    \item a poset $(\mathcal{A}, \sqsupseteq)$ with concretization and abstraction function,
    \item effective and sound abstract operators,
    \item an iteration strategy.
\end{enumerate}


\subsection{Non-numerical abstract domains}

\subsubsection{Memory abstraction}
% see https://www-apr.lip6.fr/~mine/enseignement/mpri/current/05-mpri-0-mem.pdf
\add{ memory models }

\add{ pointer abstractions }

\add{ separation logic in abstraction }

\add{ shape abstract domains }

\subsubsection{Shape analysis}

\add{ three value logic }

\add{ combination with value abstraction }

\subsubsection{String abstractions}

%Later the technique was extended by complementation \cite{Cortesi1995}.

%\section{Abstraction Framework}
%See handbook of model checkin chap. 13

%\subsection{Soundness}
% A principal requirement for any abstraction framework is that it is sound.


% see 13.4.1.1 Specific Abstractions: Examples
% 13.4.2 Additional Reading

% TODO Symbolic Methods to Enhance the Precision of Numerical Abstract Domains

\subsection{Real-world programs dedicated abstractions}

\add{ floats abstraction }

\add{ An Abstract Domain for Bit-Vector Inequalities }

\add{ Sound Bit-Precise Numerical Domains \cite{Sharma2017} }

\subsection{Domain Refinement}
\label{sec:domainrefinement}

\begin{marginfigure}%
    \centering
\resizebox{\textwidth}{!}{
    \begin{tikzpicture}[node distance=2em]
        \node [align=center] (con) {$\mathcal{P}(\mathbb{Z})$ \\ \textsf{concrete domain}};
        \node [align=center, below = of con] (ci) {\textsf{intervals and congruences}};
        \node [align=center, below right = of ci] (co) {\{$a\mathbb{Z} + b\} \cup \{\bot\}$ \\ \textsf{congruences}};
        \node [align=center, below left = of ci] (i) {$\{[a,b]\} \cup \{\bot\}$ \\ \textsf{intervals}};
        \node [align=center, below = of co] (c) {$\mathbb{Z} \cup \{\top, \bot\}$ \\ \textsf{constants}};
        \node [align=center, below = of i] (s) {$\{0, +, -, \top, \bot\}$ \\ \textsf{signs}};
        \node [align=center, below = 11em of ci] (n) {$\{0, \top, \bot\}$ \\ \textsf{nullness}};
        \node [align=center, below = of n] (d) {$\{\top, \bot\}$ \\ \textsf{dead code}};
        \node [align=center, below = of d] (t) {$\{\top\}$ \\ \textsf{no information}};

        \draw [thin] (con) -- (ci) -- (i)  -- (s) -- (n) -- (d) -- (t);
        \draw [thin] (ci) -- (co) -- (c)  -- (n);
        \draw [thin] (c) -- (i);
    \end{tikzpicture}
}
    \caption{Lattice of non-relational abstract domains.}
    \label{fig:latticeabs}%
\end{marginfigure}%


So far, we have seen abstract domains with various expressiveness, cost, and
precision. However, in many cases, we want to start with the simplest domains
and refine them automatically on the fly when a spurious counter-example is
encountered. The first approach of refinement is based on the observation that
also abstraction form a lattice (see \autoref{fig:latticeabs}). In the lattice
of domains we can refine two domains by a meet operation on them -- also known
as reduced product construction \cite{Toubhans2013, Cousot2011b}.

Many abstract domains are not closed under union. Usually, the approximation
induced by abstract unions is a major source of precision loss. Disjunctive
completion methods address this problem \cite{File1999}. This complementation
introduces to abstraction \domain{} all those concrete disjunctions of elements
of \domain{} initially missing in \domain{}. The resulting domain is the most
abstract domain that is exact on disjunctions of abstract properties in
\domain{}.  The usual usecase of abstract dijunctions is to merge multiple paths
of program.  Similarly complementation supplements missing complements to the
abstraction \cite{Cortesi1995}.

On the other hand, the fixpoint completion is a program dependent refinement,
which improves the abstract domain by introducing all the properties which
would make the fixpoint iteration imprecise \cite{Giacobazzi2000,
Giacobazzi2001}.

\add{ todo investigate more approaches }

\subsection{ Tools }

\add{ ASTREE, SeaHorn }

\section{Abstraction based techniques}
\label{sec:techniques}

\subsection{Abstraction-based model checking}

\add{ Abstraction is used in model checking to make model smaller. In general we are talking about overapproximating abstraction, wham mean that if an abstract model is correct the concrete model is also. However there is no guarantee that there is an error when ... }

% reachability algorithm

\subsection{Refinement}

\section{Counterexample-guided abstraction refinement}

% algorithm

% picture

\section{Lazy abstraction}

\section{Eager Abstraction}

\section{Symbolic Abstraction}

\section{Interpolation}

% policy iteration
