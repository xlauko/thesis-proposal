\chapter{Introduction}
\label{ch:intro}

% Stručné uvedení do problematiky, motivace, přehled dalších kapitol, 2 - 3 strany.

The contemporary research in computer science is increasingly characterized by
an overwhelming number of new techniques, tools, and algorithms.  However, many
of them are not designed with reusability in mind.  This trend causes
researchers to reimplement many of algorithms by themselves if they want to
extend existing work.  The reimplementation does not just require extra time
resources, but also produces hardly comparable results with previous approaches
because the evaluation is becoming implementation-dependent.

The computer-aided verification community is not an exception.
According to the classification of used techniques in recent \svcomp \cite{SVCOMP2019}, many tools utilize a similar set of core techniques (see \autoref{tab:svcomp}).

%\add{ verification motivation? }

The prominent role of presented approaches is taken by \emph{abstraction} related techniques, concretely \cegar
\cite{Clarke20}, predicate abstraction \cite{Flanagan02}, shape analysis \cite{Yang2008}, lazy abstraction \cite{Henzinger2002}, or symbolic data representation \cite{King76,Burch1990,Majumdar2018}.
Even though these techniques share similar ideas each of the tools ship with their own implementation.
Moreover, the design of these algorithms is not well suited for reusability.

% besides sv-comp abstraction is widely adopted to tackle .... (google scholar citation)

On the other hand, we may also find examples of reusable techniques.  One such
example is the utilization of \smt logics and solvers in the symbolic
verification.  With widely adopted format SMT-LIB \cite{Barrett2010}, \smt
solvers are easily used out-of-the-box \cite{Dutertre2006, DeMoura2008, Barrett2011, Corzilius2015}.

In general, a goal of computer-aided verification is to decide whether a system
under test satisfies a given specification.  Nevertheless, the verification
task is undecidable; many verifiers can reason about complex systems. This
would not be possible without the abstraction of system behaviors.

% TODO what i will present in my thesis

% In order to contribute to the solution of the software reliability problem, tools have been designed to analyze statically the run-time
% behavior of programs. Because the correctness problem is undecidable,
% some form of approximation is needed. The purpose of abstract interpretation is to formalize this idea of approximation. We illustrate informally
% the application of abstraction to the semantics of programming languages
% as well as to static program analysis. [Abstract Interpretation Based Formal Methods and Future Challenges]

\begin{table}[h]

	\centering
    \resizebox{\textwidth}{!}{%
    \newcolumntype{C}{>{\sffamily}l}
    \begin{tabular}{C |  g c g c g c g c g c g c g c g c g c}
        \textsc{participant}
        & \rotatebox{90}{\cegar}
        & \rotatebox{90}{Predicate Abstraction}
        & \rotatebox{90}{Symbolic Execution}
        & \rotatebox{90}{Bounded Model Checking}
        & \rotatebox{90}{k-Induction}
        & \rotatebox{90}{Property-Directed Reach.}
        & \rotatebox{90}{Explicit-Value Analysis}
        & \rotatebox{90}{Numeric. Interval Analysis}
        & \rotatebox{90}{Shape Analysis}
        & \rotatebox{90}{Separation Logic}
        & \rotatebox{90}{Bit-Precise Analysis}
        & \rotatebox{90}{ARG-Based Analysis}
        & \rotatebox{90}{Lazy Abstraction}
        & \rotatebox{90}{Interpolation}
        & \rotatebox{90}{Automata-Based Analysis}
        & \rotatebox{90}{Concurrency Support}
        & \rotatebox{90}{Ranking Functions}
        & \rotatebox{90}{Evolutionary Algorithms}
        \\
 	\hline
    2LS             &  &  &  & \cm & \cm &  &  & \cm &  &  & \cm &  &  &  &  &  & \cm &  \\
    AProVE          &  &  & \cm &  &  &  & \cm & \cm &  & \cm & \cm &  &  &  &  &  & \cm &  \\
    \hline
    CBMC            &  &  &  & \cm &  &  &  &  &  &  & \cm &  &  &  &  & \cm &  &  \\
    CBMC-Path       &  &  &  & \cm &  &  &  &  &  &  & \cm &  &  &  &  & \cm &  &  \\
    \hline
    CPA-BAM-BnB     & \cm & \cm &  &  &  &  & \cm &  &  &  & \cm & \cm & \cm & \cm &  &  &  &  \\
    CPA-Lockator    & \cm & \cm &  &  &  &  & \cm &  &  &  & \cm & \cm & \cm & \cm &  & \cm &  &  \\
    \hline
    CPA-Seq         & \cm  & \cm &  & \cm & \cm &  & \cm & \cm & \cm &  & \cm & \cm & \cm & \cm &  & \cm & \cm &  \\
    DepthK          &  &  &  & \cm & \cm &  &  &  &  &  & \cm &  &  &  &  & \cm &  &  \\
    \hline
    DIVINE-explicit &  &  &  &  &  &  & \cm &  &  &  & \cm &  &  &  &  & \cm &  &  \\
    DIVINE-SMT      &  &  &  &  &  &  & \cm &  &  &  & \cm &  &  &  &  & \cm &  &  \\
    \hline
    ESBMC-kind      &  &  &  & \cm & \cm &  &  &  &  &  & \cm &  &  &  &  & \cm &  &  \\
    JayHorn         & \cm & \cm &  &  &  & \cm &  & \cm &  &  &  &  & \cm & \cm &  &  &  &  \\
    \hline
    JBMC            &  &  &  & \cm &  &  &  &  &  &  & \cm &  &  &  &  & \cm &  &  \\
    JPF             &  &  &  & \cm &  &  & \cm & \cm &  &  & \cm &  &  &  &  &  &  &  \\
    \hline
    Lazy-CSeq       &  &  &  & \cm &  &  &  &  &  &  & \cm &  &  &  &  & \cm  &  &  \\
    Map2Check       &  &  &  & \cm &  &  &  &  &  &  & \cm &  &  &  &  &  &  &  \\
    \hline
    PeSCo           & \cm & \cm &  & \cm & \cm &  & \cm & \cm & \cm &  & \cm & \cm & \cm & \cm &  & \cm & \cm &  \\
    Pinaka          &  &  & \cm & \cm &  &  &  &  &  &  & \cm &  &  &  &  &  &  &  \\
    \hline
    PredatorHP      &  &  &  &  &  &  &  &  & \cm &  &  &  &  &  &  &  &  &  \\
    Skink           & \cm &  &  &  &  &  & \cm &  &  &  &  &  &  & \cm & \cm &  &  &  \\
    \hline
    Smack           & \cm &  &  & \cm &  & \cm &  &  &  &  & \cm &  & \cm &  &  & \cm &  &  \\
    SPF             &  &  & \cm &  &  &  &  &  & \cm &  &  &  &  &  &  & \cm &  &  \\
    \hline
    Symbiotic       &  &  & \cm &  &  &  &  & \cm &  &  & \cm &  &  &  &  &  &  &  \\
    UAutomizer      & \cm & \cm &  &  &  &  &  &  &  &  & \cm &  & \cm & \cm & \cm &  & \cm &  \\
	\hline
    UKojak          & \cm & \cm &  &  &  &  &  &  &  &  & \cm &  & \cm & \cm &  &  &  &  \\
    UTaipan         & \cm & \cm &  &  &  &  &  &  &  &  & \cm &  & \cm & \cm & \cm &  &  &  \\
	\hline
    VeriAbs         & \cm &  &  & \cm & \cm &  & \cm & \cm &  &  &  &  &  &  &  &  &  &  \\
    VeriFuzz        &  &  &  & \cm &  &  &  & \cm &  &  &  &  &  &  &  &  &  & \cm \\
	\hline
    VIAP            &  &  &  &  &  &  &  &  &  &  &  &  &  &  &  &  &  &  \\
    Yogar-CBMC      & \cm &  &  & \cm &  &  &  &  &  &  & \cm &  & \cm &  &  & \cm &  &  \\
	\hline
    Yogar-CBMC-Par. & \cm &  &  & \cm &  &  &  &  &  &  & \cm &  & \cm &  &  & \cm &  &  \\
	\end{tabular}
    }

\caption{Techniques that the \svcomp 2019 candidates offer \cite{SVCOMP2019}.}
\label{tab:svcomp}
\end{table}

%\change{ REWRITE: Notwithstanding the amount of work required many tools reinvent the wheel and adapt the existing algorithms.
%The main success of abovementioned techniques is in their indispensability.
%Since program verification is generally undecidable, the automatic computer-aided verification tools rely on the approximation techniques.
%Either they provide only partial algorithms, requires human interaction (in the case of theorem provers) or consider a restricted form of programs and their properties. }
%\add{ add introduction to the rest of introduction }
%\add{ introduce abstractions }

The model on which the verifier operates introduces the first approximation of the real environment.
In general, it would be impractical if not almost impossible to verify the whole environment, starting from the underlying hardware and ending with an operating system and user interactions.
Hence, verifiers assume the correctness of underlying layers of the program and focus only on the system undergoing test.
Another approximation made by verifiers is on the level of program semantics.  When verifying a program (for example in C), one can decide whether to perform atomic semantic steps by program lines or based on the compiled assembly code. Whereas the first may preserve a piece of information about the behavior of the program, the letter is closer to the real execution.
The last approximation of our concern is data abstraction that is used to represent non-deterministic values in the program and tackles the state-space explosion problem.

In this thesis proposal, we present essential as well as contemporary abstraction and symbolic techniques. We identify their common specifics and present a potential design for reausable
abstraction across tools.

The rest of the proposal is organized as follows: \autoref{ch:state} summarizes the state of the art of approximation techniques, the \autoref{sec:preliminaries} present a common notation used in following chapters, followed by introduction to an abstraction (\autoref{sec:abstraction}) and summary of abstract domains \autoref{sec:domains}. The second half of \autoref{ch:state} investigates abstraction-based techniques used in computer-aided verification (\autoref{sec:techniques}).

The following \autoref{ch:aim} examines the presented approaches and states the
challenges of reusable abstraction design. We introduce a notion of a
transformation-based abstraction, which is one of the possible solutions for
the stated objectives. Further, we present possible directions for the thesis
in \autoref{sec:objectives}.

The last chapter (\autoref{ch:results}) presents already achieved results in
transformation-based approach.  \autoref{sec:symbolic} investigates how we
have repurposed a symbolic computation for explicit-based model checking
\cite{Lauko2019Sym, Lauko2018SymComp}. Further in \autoref{sec:string}, we
present a technique for string and parametrizable abstract domains which were
integrated into the transformation-based approach [TODO].  In the end
(\autoref{sec:publications}), the chapter lists my already published papers
related to abstraction techniques.
