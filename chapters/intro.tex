\chapter{Introduction}
\label{ch:intro}

% Stručné uvedení do problematiky, motivace, přehled dalších kapitol, 2 - 3 strany.

The contemporary research in computer science is starting to be snowed under a lot of new techniques, tools, and algorithms.
However, many of them are not designed with reusability in mind.
This trend causes researchers to reimplement many of algorithms by themselves if they want extend existing work.
The reimplementation does not just cost extra time resources but also produce hardly comparable results with previous approaches.

The computer verification community is not an exception.
According to the classification of used techniques in recent \svcomp \cite{SVCOMP2019}, many tools utilize a similar set of core techniques.\sidenote{The techniques used by tools in \svcomp 2019 are summarized in table \autoref{tab:svcomp}.}

The prominent role of presented approaches is taken by \emph{abstraction} related techniques, concretely \cegar, predicate abstraction, shape analysis, lazy abstraction or symbolic data representation.
Even though these techniques share the similar ideas each of the tools ships its own implementation of them.

% besides sv-comp abstraction is widely adopted to tackle .... (google scholar citation)

On the other hand, we may find also examples of reusable techniques.
One such example is a utilization of \smt logics and solvers in the symbolic verification.
With widely adopted format SMT-LIB \cite{Barrett2010}, \smt solvers are easily used out-of-the-box [TODO].
\add{extend}
% TODO what i will present in my thesis

In this thesis proposal we present a survey of current abstraction techniques and  present a technique how to generalize them to be reusable across \llvm based tools.

\add{propose a generalized}

\add{llvm intro}

% In order to contribute to the solution of the software reliability problem, tools have been designed to analyze statically the run-time
% behavior of programs. Because the correctness problem is undecidable,
% some form of approximation is needed. The purpose of abstract interpretation is to formalize this idea of approximation. We illustrate informally
% the application of abstraction to the semantics of programming languages
% as well as to static program analysis. [Abstract Interpretation Based Formal Methods and Future Challenges]

\begin{table}[h]

	\centering
    \resizebox{\textwidth}{!}{%
    \begin{tabular}{l | llllllllllllllllll}
    \textsc{participant} &  &  &  &  &  &  &  & &  &  &  &  &  &  &  &  &  &  \\
 	\hline
    &  &  &  &  &  &  &  &  &  &  &  &  &  &  &  &  &  &  \\
    &  &  &  &  &  &  &  &  &  &  &  &  &  &  &  &  &  &  \\
    &  &  &  &  &  &  &  &  &  &  &  &  &  &  &  &  &  &  \\
    &  &  &  &  &  &  &  &  &  &  &  &  &  &  &  &  &  &  \\
    &  &  &  &  &  &  &  &  &  &  &  &  &  &  &  &  &  &  \\
    &  &  &  &  &  &  &  &  &  &  &  &  &  &  &  &  &  &  \\
    &  &  &  &  &  &  &  &  &  &  &  &  &  &  &  &  &  &  \\
    &  &  &  &  &  &  &  &  &  &  &  &  &  &  &  &  &  &  \\
    &  &  &  &  &  &  &  &  &  &  &  &  &  &  &  &  &  &  \\
    &  &  &  &  &  &  &  &  &  &  &  &  &  &  &  &  &  &  \\
    &  &  &  &  &  &  &  &  &  &  &  &  &  &  &  &  &  &  \\
    &  &  &  &  &  &  &  &  &  &  &  &  &  &  &  &  &  &  \\
    &  &  &  &  &  &  &  &  &  &  &  &  &  &  &  &  &  &  \\
    &  &  &  &  &  &  &  &  &  &  &  &  &  &  &  &  &  &  \\
    &  &  &  &  &  &  &  &  &  &  &  &  &  &  &  &  &  &  \\
    &  &  &  &  &  &  &  &  &  &  &  &  &  &  &  &  &  &  \\
    &  &  &  &  &  &  &  &  &  &  &  &  &  &  &  &  &  &  \\
    &  &  &  &  &  &  &  &  &  &  &  &  &  &  &  &  &  &  \\
    &  &  &  &  &  &  &  &  &  &  &  &  &  &  &  &  &  &  \\
    &  &  &  &  &  &  &  &  &  &  &  &  &  &  &  &  &  &  \\
    &  &  &  &  &  &  &  &  &  &  &  &  &  &  &  &  &  &  \\
    &  &  &  &  &  &  &  &  &  &  &  &  &  &  &  &  &  &  \\
    &  &  &  &  &  &  &  &  &  &  &  &  &  &  &  &  &  &  \\
    &  &  &  &  &  &  &  &  &  &  &  &  &  &  &  &  &  &  \\
	\hline
	\end{tabular}
    }

\caption{Technologies and features that the competition candidates offer.}
\label{tab:svcomp}
\end{table}

\add{larger introduction to techniques - abstraction motivation}



The thesis proposal is organized as follows: \autoref{ch:state} summarizes the state of the art and is divided into {\color{red}x} parts. \autoref{ch:results}
In the following chapter

\add{chapter description}
