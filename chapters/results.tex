\chapter{achieved results}
\label{ch:results}

% Povinné pro rigorózní práci, 3-5 stran.

In my research, I focus on the design of self-contained abstraction for program
verification. The main objective of this design is to determine how to refactor
abstraction out of dynamic analysis and still provide sufficient information
for program state-space exploration. In my work, I propose to let the program
under test take care of abstraction and maintain abstract data. This can be
achieved by instrumentation of abstract operations directly into the program.
Hence, the verifier does not need to perform abstract interpretation, but it
can only perform a~concrete interpretation of the program and obtain abstract
data from the program memory if it needs them.

In the previous chapter, I have stated multiple alternatives of how much of the
abstract interpretation is instrumented into the program and retained in the
interpreter.  As a reasonable boundary for reusable abstraction, I have
selected a design where metadata and decision procedure are part of the
execution engine (see \autoref{fig:bounds}). This choice balances an effort
required to integrate the transformation-based technique and load on execution
engine, where the instrumented code needs to be interpreted.

Up until now, I have focused on the general design of trans\-for\-mation-based
abstraction without refinement (\emph{trivially constrainable domains}). The
main concern of the first prototype was to design a general-purpose analysis
for instrumentation. The goal of these transformations was to allow coexistence
of concrete and abstract values in a single program. In this way, I was able to
extend model checker \divine~\cite{Divine17} with symbolic
computation~\cite{Lauko2019Sym, Lauko2018SymComp}.

All analyses presented in my up-to-date work are implemented in a~tool named
\lart{} -- \llvm abstraction and refinement tool, that is a~part of the \divine
project. \lart is based on top of the open-source \llvm compiler
infrastructure. It allows \lart to easily integrate program analyses,
transformations, and other tools that target \llvm.

\section{Transformation-based abstraction}
\label{sec:symbolic}

\subsection{Abstraction design}

The overall symbolic abstraction, depicted in \autoref{fig:design}, consists of
two parts: preprocessing (\emph{static analysis}), which performs
instrumentation of abstraction into the program; and verification
(\emph{dynamic analysis}) that explores the state space of the system under
test.

The instrumentation performs data-flow analysis of non-de\-ter\-ministic values and
replaces operations on them by their abstract counterparts. The instrumented
program is subsequently linked with a library that implements abstract
operations and provides definitions of abstract data representation.

During verification, virtual machine (\textsc{vm}) executes the abs\-trac\-ted
program and generates state space for a model checking algorithm (\textsc{mc}).
The model checking algorithm obtains symbolic data and path condition directly
from program memory, whenever it needs to decide whether the state is feasible
or is equal to some previously visited state. An external \textsc{smt} solver solves
feasibility and equality queries. Notice that the virtual machine has no knowledge
about abstract values present in the program.
However, it needs to recognize what parts of the memory contains abstract data,
because they can not be compared explicitly when the model checker is
determining equality of states. Therefore, the virtual machine keeps metadata
that denotes what parts of memory hold abstract values.

\begin{figure}
    \centering
    \input{img/approaches.tikz}
    \caption{Components of symbolic computation in \divine.}
    \label{fig:design}
\end{figure}

\subsection{Term domain}

The abstract domain used for the symbolic computation was inspired by the
domain of uninterpreted functions~\cite{Gange2016} -- we call it a term domain.
In the term domain, arithmetic operations are composed into term trees (data
definitions) that represent relations between ground variables and constants,
whereas the branching conditions extends the global path condition.  For
further details, see papers in the attachments of the thesis
proposal~\cite{Lauko2018SymComp, Lauko2019Sym}.

\section{String abstraction}
\label{sec:string}

My following publication has focused on the extension of the
trans\-for\-ma\-tion-based approach to non-scalar domains (i.e., aggregate
domains).  An example of such a domain is a string domain. In cooperation with
Martina Olliaro, I have adapted her string domain
(\textsc{mstring})~\cite{Olliaro2018} as \lart domain. A \lart domain is a C or
C++ library that provides definitions of abstract values and implementation of
supported operations on them.

\textsc{Mstring} domain represents C strings by segmentation abstract domain
that keeps track of terminal zeros in the underlying array. It is
a~parametrizable domain in the means of array segments\sidenote{Array segment
represents a consecutive area of the same characters.} length and array
content.  Hence, we have repurposed a symbolic domain from previous work to
abstract lengths of string segments.

To support aggregate domains in \lart, I have adapted its data-flow analysis to
carry multiple domains at the same time, as well as to be able to analyze
manipulations with \llvm arrays. Moreover, I have extended the instrumentation
to support abstraction of larger operations than just \llvm instructions --
these occur in C string manipulating programs in the form of library functions
(\texttt{strlen}, \texttt{strcmp}, \texttt{strcat}). The transformation-based
string abstraction is further presented in the attached paper
~\cite{Lauko2019String}.

\section{Publications}
\label{sec:publications}

\begin{itemize}

\item \textbf{\emph{Symbolic Computation via Program Transformation}}, by Lauko
H., P. Ročkai and J. Barnat.  In: Theoretical Aspects of Computing -- ICTAC 2018 --
15th International Colloquium, Stellenbosch, South Africa, October 16-19, 2018,
Proceedings, pp. 313-332~\cite{Lauko2018SymComp}

I am the main author of the text of the paper. I have also implemented the
transformation-based abstraction and conducted all the experiments.

\item \emph{\textbf{Extending \textsf{DIVINE} with Symbolic Verification Using SMT}},
by Lauko H., V. Štill, P. Ročkai and J. Barnat.
In: Tools and Algorithms for the Construction and Analysis of Systems -- TACAS 2019 --
25 Years of TACAS: TOOLympics, Prague, Czech Republic, April 6-11, 2019,
Proceedings, Part III. pp. 204-208~\cite{Lauko2019Sym}

I am the primary author of the text of the paper, as well as of the
abstraction implementation. The implementation included an adaptation of \divine
abstraction support to handle \svcomp benchmarks.

\item \textbf{\emph{String Abstraction for Model Checking of C Programs}},
by Cortesi A., H. Lauko, M. Olliaro and P. Ročkai.
        In: Model Checking Software -- SPIN 2019 -- 26th International Symposium~\cite{Lauko2019String}

I am the author of half of the paper content: primarily part concerning
transformation-based abstraction and experiments. I also have concluded
experiments and implemented an abstract string domain and accompanying
\lart analyses.

\end{itemize}

\section{Other publications}

\begin{itemize}

\item \textbf{\emph{Model Checking of C and C++ with DIVINE 4}},
by Baranová Z., J. Barnat, K. Kejstová, T. Kučera, H. Lauko, J. Mrázek, P. Ročkai and V. Štill.
In: Automated Technology for Verification and Analysis -- ATVA 2017 --
15th International Symposium, Pune, India, October 3-6, 2017,
Proceedings, pp. 201-207~\cite{Divine17}

I am a responsible person for implementation of abstraction and symbolic
computation support in \divine 4.

\item \emph{\textbf{Optimizing and Caching SMT Queries in SymDIVINE}},
by Mrázek J., M. Jonáš, V. Štill, H. Lauko and J. Barnat.
In: Tools and Algorithms for the Construction and Analysis of Systems -- TACAS 2017 --
23rd International Conference, Uppsala, Sweden, April 22-29, 2017,
Proceedings, Part II. pp. 390-393~\cite{Mrazek2017}

I wrote parts of the paper and I have helped with a tool preparation for the
\svcomp competition.

\item \textbf{\emph{\textsf{SymDIVINE}: Tool for Control-Explicit Data-Symbolic State \\ Space Exploration}}
by Mrázek J., P. Bauch, H. Lauko and J. Barnat.
In: Model Checking Software -- SPIN 2016 -- 23rd International Symposium, Eindhoven, The Netherlands, April 7-8, 2016, Proceedings, pp. 208-213~\cite{Mrazek2016}

I have helped with writing of the final paper.

\end{itemize}
