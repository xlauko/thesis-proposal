\chapter{\llvm syntax}
\label{ch:llvm}

\section*{Types}
\begin{grammar}

<float> ::= "half" | "float" | "double" | "fp80" | "fp128"

<type> ::= "void" | "i"<sz> | <float> | <type>* | "[" <sz> $\times$ <type> "]" | \{ $<type>^n$ \} | "id"

\end{grammar}

\noindent
where $sz \in \mathbb{N}^+$ is a natural number. \llvm recognize
integer types denoted by \texttt{i} and their bit-width $sz$, floating point
types, pointer types denoted by \texttt{*}, array types of predetermined size,
heterogeneous structure types wrapped in curly braces, and value identifiers.
\llvm does not have a dedicated boolean type, instead it uses a one bit-long
integer type \texttt{i1}.

\section*{Operations}

In our restricted \llvm, we recognize binary arithmetic operations on integers
and floating points. Since \llvm integer types do not preserve signedness,
operations that depend on the sign (e.g., division \texttt{div} and reminder
after division \texttt{rem}) have multiple variations for unsigned and signed
integers.

\begin{grammar}
<bop> ::= "add" | "sub" | "mul" | "udiv" | "sdiv" | "urem" | "srem" \alt
          "shl" | "lshr" | "ashr" | "and" | "or" | "xor"

<fbop> ::= "fadd" | "fsub" | "fmul" | "fdiv" | "frem"
\end{grammar}

\noindent
To extend or truncate bit-width of scalar values \llvm utilizes following
instructions, where the extension can preserve the sign of value (the most
significant bit) by \texttt{sext} or extend the value with zeros \texttt{zext}:

\begin{grammar}
<eop>  ::= "zext" | "sext" | "fpext"

<trop> ::= "trunc" | "fptrunc"
\end{grammar}

\noindent
Cast instructions perform the transformation between integer, floating-point,
and pointer types, as well as between various pointer types by \texttt{bitcast}:

\begin{grammar}
<cast>  ::= "bitcast" | "ptrtoint" | "inttoptr" | "fptoui" | "fptosi" \alt
           "uitofp"  | "sitofp"
\end{grammar}

\noindent
Relations between \llvm values are determined by comparison operators. They
also contains of singed and unsigned versions for integer values, as well as
order and unsigned versions for floating-point values. They denote equality
\texttt{eq}, non-equality \texttt{ne}, greater than relation \texttt{gt},
greater or equal relation \texttt{ge}, less than relation \texttt{lt} and less
than or equal relation \texttt{le}.

\begin{grammar}
<irel> ::= "eq" | "ne" | "ugt" | "uge" | "ult" | "ule" | "sgt" | "sge" | "slt" | "sle"

<frel> ::= "false" | "oeq" | "ogt" | "oge" | "olt" | "ole" | "one" | "ord" | "ueq" | "ugt" | "uge"
    \alt "ult" | "ule" | "une" | "uno" | "true"
\end{grammar}

\section*{Values}

In \llvm program we allow constant values, dynamic values denoted by
identifiers, global variables, function definitions and declarations:

\begin{grammar}
<cnst> ::= "i"<sz> "Int" | <float> "Float" | <type>* "null" | <type> "undef" |
    \alt <type> "zeroinitializer"

<val> ::= <id> | <cnst>

<arg> ::= <type> <id>

<glob> ::= "global" <type> <cnst>
    \alt "define" <type> <id>( $<arg>^n$ ) \{ $<block>^m$ \}
    \alt "declare" <type> <id>( $<arg>^n$ )
\end{grammar}

\section*{Control Flow}

Control flow graph of \llvm functions consists of labeled basic blocks, where
each starts with a sequence of phi node instructions (merge points of \ssa
form), and sequence of instructions terminated with a terminal instruction.
Terminal instruction either branch via conditional/uncon\-di\-tio\-nal branch to the
subsequent basic block or they return from the function. Alternatively, an
unreachable location is denoted by special instruction.

\begin{grammar}

<block> ::= $label$ $<phi>^n$ $<cmd>^m$ <term>

<phi> ::= <id> = "phi" <type> [$<val>$, $label$]$^n$

<term> ::= "br" <val> $label$ $label$  | "br" $label$
    \alt "ret" <type> <val> | "ret void"
    \alt "unreachable"
\end{grammar}

\section*{Commands}

Ids identify results of \llvm commands, which are either \ssa instructions
(binary operations, casts, or relational operations), memory access operations, or
function calls. A special instruction getelementptr is used to compute offset
from a pointer.

\begin{grammar}
<cmd> ::= <id> = <bop> "i"<sz> <val> <val>
    \alt <id> = <fbop> <float> <val> <val>
    \alt <id> = "load" <type>* <val>
    \alt "store" <type> <val> <val>
    \alt <id> = "malloc" <type>
    \alt "free" <type>* <val>
    \alt <id> = "alloca" <type>
    \alt <id> = <cast> <type> <val> "to" <type>
    \alt <id> = "icmp" <irel> <type> <val> <val>
    \alt <id> = "fcmp" <frel> <type> <val> <val>
    \alt call <type> <id> $<val>^n$ | <id> = call <type> <id> $<val>^n$
    \alt <id> = getelementptr <type>* <val> $<val>^n$
\end{grammar}

\noindent
The semantics of these instructions is well described in the \llvm language
reference manual\footnote{\url{https://llvm.org/docs/LangRef.html}}, and formalized in~\cite{Zhao2012}.
